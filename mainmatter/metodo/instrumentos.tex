\section{INSTRUMENTOS}\label{ins}
El conjunto de instrumentos se elabor\'o con los  siguientes recursos:
\begin{itemize}
\item Humanos
	\begin{itemize}
	\item Profesionales -i.e. Entrenadores-.
	\item Atletas
	\item Investigador
	\end{itemize}
\item No humanos
	\begin{itemize}
	\item Mesa de soporte de altura de 0.70 metros.
	\item Cable de extensi\'on el\'ectrico de 2.00 metros. 
	\item Computadora p\'ortatil
		\begin{itemize}
		\item Conector de carga de alimentaci\'on
		\end{itemize}
	\item Sensor Kinect
		\begin{itemize}
		\item Adaptador del Kinect
		\end{itemize}
	\end{itemize}
\end{itemize}
Estos recursos permitier
\subsection{Formulario de registro de movimiento} \label{ins:frmMov}
Este formulario se adjunta en anexos (ver figura \ref{fig:frmWhiteMov}), la cual tiene como objetivo describir el movimiento de cada equipo deportivo. Por otra parte, el formulario esta compuesto por las siguientes incisos:
\begin{itemize}
	\item \textbf{Nombre del movimiento:} Nombre que se identifica en gu\'ias deportivas o de salud.
	\item \textbf{Descripci\'on del movimiento: } Contesta la pregunta: ?`Qu\'e es el movimiento?
	\item \textbf{Movimiento unilateral:} S\'i es afirmativo, el movimiento trabaja con solo una parte del cuerpo (Izquierda o derecha, ejemplo una patada), en caso contrario el movimiento trabajo con todas las partes del cuerpo (Izquierda y derecha, ejemplo un salto).
	\item \textbf{Partes del cuerpo ignorada:} M\'ultiples respuestas, que identifica las articulaciones ignoradas -i.e. Debido que no es informaci\'on relevante para el movimiento-:
	\begin{itemize}
		\item \textbf{Brazo derecho o izquierdo:} Ignora las siguientes articulaciones (seg\'un su lado):Pulgar, dedo del medio, mano, codo, hombro, centro de los hombros, cuello, cabeza y espalda.
		\item \textbf{Cuerpo inferior:} Ignora las siguientes articulaciones (en ambos lados): centro de cadera, caderas, Rodillas, Tobillos y p\'ies.
	\end{itemize}
		\item \textbf{N\'umero de pasos:} Cantidad de pasos del movimiento.
		\item \textbf{Offset del paso:} Cantidad que identifica la etiqueta del movimiento.
		\item \textbf{Valor de identificaci\'on:} Describe el valor del rango general de un paso.
		\item \textbf{Detalle por paso:} Es  importante conocer:
			\begin{itemize}
		\item \textbf{Paso:} N\'umero \'unico que identifica el paso.
		\item \textbf{Diagrama:} Imagen visual del paso -i.e. Seguimiento de esqueleto-.
		\item \textbf{Descripci\'on del paso:} Responde la pregunta: ?`Cu\'al es la postura del cuerpo durante ese paso?
		\item \textbf{Etiqueta:} N\'umero \'unico que identifica el paso a partir de la probabilidad de movimiento.
	\item \textbf{Rango de identificaci\'on:} Rango m\'aximo que identifica el paso dentro de la probabilidad del movimiento.
	\end{itemize}
\end{itemize}
\subsection{Formulario de registro de rutina} \label{ins:frmRout}
Formulario adjuntado en anexos (Ver figura \ref{fig:frmWhiteRout}), que tiene como funci\'on identificar los ejercicios de calentamiento  previamente a realizar el movimiento seleccionado por cada equipo deportivo, adem\'as de estandarizar el n\'umero de repeticiones del movimiento por cada atleta, a partir de los siguientes puntos:
\begin{itemize}
	\item \textbf{Descripci\'on del calentamiento:} Detalla el tipo de calentamiento -e.g. Calentamiento con tu propio peso, calentamiento dentro de un ambiente, calentamiento a partir de objetos-.
	\item \textbf{Movimientos:} Enlista los movimientos que se realiza durante el calentamiento (Validado por las unidades de an\'alisis, secci\'on  \ref{sj:ua}).
	\item \textbf{Series:} Cantidad de veces que debe realizar un atleta, un conjunto de repeticiones del movimiento.
	\item \textbf{Repeticiones:} Cantidad de repeticiones del movimiento.
	\item \textbf{Tiempo:} duraci\'on del calentamiento.
	\item \textbf{Im\'agenes:} Fotograf\'ias tomadas durante el entrenamiento de cada equipo deportivo.
	\item \textbf{Nombre del movimiento:} Movimiento seleccionado por cada equipo deportivo  (secci\'on  \ref{ins:frmMov}).
	\item \textbf{Rutina:} Describe el tipo de rutina para la captura de datos:
	\begin{itemize}
		\item \textbf{For time:} M\'axima cantidad de repeticiones del movimiento, durante un tiempo establecido.
		\item \textbf{Escaleras:} N\'umeros de repeticiones establecida, separada por series.
	\end{itemize}	
\end{itemize}
\subsection{Interfaces de usuarios} \label{ins:UI}
Aplicaciones que interact\'ua con el usuario para la recoleccio\'n de datos, en donde se trabaja con  tres tipos:
\subsubsection{Presentation Foundation (WPF)}
\label{ins:UI:wpf}
El desarrollador, \citeA{wpf2019}, menciona que este tipo de interfaz permite crear aplicaciones de escritorio con el framework .NET, la cual es soportado desde Windows XP hasta la \'ultima versi\'on de Windows -i.e. Windows 10-. Por lo tanto, en el presente proyecto se utiliz\'o este tipo de interfaz para la construci\'on del seguimiento de esqueleto, generada por los siguientes Kit de desarrollo de software:
\begin{itemize}
	\item \textbf{Windows inputs} Herramienta que permite crear elementos de un formulario -e.g. Botones, checkbox, textbox-.
	\item \textbf{Windows Media} Herramientas que permite renderizar el seguimiento de esqueleto en tiempo real a partir de pinceles, colores, formas y dibujos.
	\item \textbf{Windows Threading} Herramientas que permite crear temporizadores -i.e timer- para la renderizaci\'on del seguimiento del esqueleto en un per\'iodo del tiempo -i.e. 30 fps-.
	\item \textbf{Kinect} Herramientas que permite acceder a las funcionalidades del Kinect, tales:
	\begin{itemize}
	\item \textbf{Sensor Kinect:} Accede al API del Kinect para realizar las siguientes tareas:
		\begin{itemize}
				\item \textbf{Body frame reader:} Obtiene la informaci\'on del seguimiento del esqueleto -i.e. Esqueletos reconocidos con sus respectivas articulaciones-.
				\item \textbf{Estado del sensor:} Activo, Pausa, No detectado e inactivo.
				\item \textbf{Datos del visual Gesture Builder:} Normalizaci\'on de datos del sensor del Kinect para el uso del modelo de machine learning.
		\end{itemize}	
			\item \textbf{Joint type:} Enumeradores que enlista las articulaciones del seguimiento del esqueleto -e.g. Manos, codos, hombros-.
			\item \textbf{Visualizaci\'on de los recursos de los frames para Visual Gesture Builder:} Interpreta la base de datos de gesturas y posiciones de un movimiento.
	\end{itemize}	 
\end{itemize}
A partir de los kits de desarrollo, se cre\'o 2 aplicaciones para la captura de informaci\'on.
\paragraph{Detecci\'on de profundidad}\mbox{} \\ \label{ins:UI:wpf:depth}
Esta aplicaci\'on tiene como objetivo recolectar la distancia correcta de profundidad entre el atleta y el Kinect, tomando en cuenta una articulaci\'on de an\'alisis, adem\'as de la altura del usuario medida desde la cabeza hasta los p\'ies, tal como se presenta en la siguiente imagen:
\begin{figure}[H]
	\caption{Interfaz gr\'afica de detecci\'on de profundidad entre el usuario y el sensor}
	\label{fig:appDepth}
	\centering
	\includegraphics[width=380px,height=200px]{graphics/appProfundidad.png} \\
	\textbf{Fuente:} Aplicaci\'on elaborada por el autor de tesis
\end{figure}
Esta interfaz se compone por 5 componentes:
\begin{enumerate}[A.]
    \item Combo box que permite seleccionar la articulaci\'on de an\'alisis.
    \item Bot\'on que empieza las funcionalidades del seguimiento del esqueleto.
    \item Bot\'on que finaliza las funcionalidades del seguimiento del esqueleto.
    \item Conjunto de paneles de control, que muestra una imagen en tiempo real del seguimiento del esqueleto, adem\'as de la altura (en metros) del usuario y la distancia de profundidad (en metros) entre el atleta y el sensor.
        \item Bot\'on que permite copiar a una hoja de observaci\'on (ver anexos, cuadro \ref{tab:obsDepth}) los siguientes datos respectivos: N\'umero de identificaci\'on de la articulaci\'on, la distancia de profundidad y la altura del usuario.
\end{enumerate}
\paragraph{Evaluaci\'on del movimiento}\mbox{} \\
Aplicaci\'on realizada por el autor del usuario, cuya funcionalidad es programar una rutina de tabata a partir de del movimiento de cada equipo deportivo, mostrado en la siguiente imagen:
\begin{figure}[H]
	\caption{Interfaz gr\'afica de evaluaci\'on de un movimiento}
	\label{fig:appEvaluate}
	\centering
	\includegraphics[width=380px,height=250px]{graphics/appEvaluacion.png} \\
	\textbf{Fuente:} Aplicaci\'on elaborada por el autor de tesis
\end{figure}
Interfaz que se divide en 13 componentes:
\begin{enumerate}[A.]
    \item Bot\'on que permite seleccionar el archivo de base de datos del reconocimiento del movimiento.
    \item Bot\'on que permite seleccionar el archivo json que contiene toda la informaci\'on respectiva del movimiento (ver anexos, c\'odigo  \ref{code:jsonMeta}).
    \item Bot\'on que permite seleccionar la direcci\'on del archivo de resultado de tabata.
    \item Bot\'on que permite activar todas las funcionalidades del Kinect.    
    \item Bot\'on que permite detener todas las funcionalidades del Kinect. 
    \item Bot\'on que permite comenzar la rutina de tabata (en tiempo real).
   \item  Text box n\'umerico que indica la cantidad de tiempos de trabajo y descanso del atleta durante su rutina de tabata.
   \item  Text box n\'umerico que se\~nala el tiempo de descanso durante su rutina.
   \item  Text box n\'umerico que muestra el tiempo de trabajo durante su rutina -i.e. Durante este tiempo, el atleta debe realizar la cantidad m\'axima de repeticiones-.
   \item  Etiqueta de articulaci\'on de \'analisis para medir la distancia de profundidad entre el atleta y el sensor.
   \item  Etiqueta de la altura  recomendada (en metros)  del sensor y el suelo.
      \item  Etiqueta de la distancia m\'inima y m\'axima de profundidad del atleta con respecto al sensor, y por otra parte indica la distancia profundidad actual del usuario y el sensor.
      \item Conjunto de paneles de controles que muestra:
      \begin{itemize}
            \item  El seguimiento de esqueleto en tiempo real.
            \item  Estado actual de la rutina: Inicio, Trabajo, Descanso, Fin.
            \item  Temporizador de cuenta regresiva del tiempo de trabajo o descanso (en segundos).
             \item  Serie actual que esta trabajando el atleta.
             \item  Contador de repeticiones de la serie actual.
              \item \'Ultimo paso ejecutado por el atleta (Comenzando desde 0).   
             \item  Valor de probabilidad del movimiento -i.e. Progreso-.
             \item  Temporizador que mide el tiempo    empleado en la rutina (en segundos).
      \end{itemize}
\end{enumerate}
Ya definido los componentes de la interfaz, se debe tomar en cuenta que al finalizar cada rutina tabata, el programa genera un archivo JSON (ver anexos, c\'odigo  \ref{code:tabata}), con los siguientes resultados:
\begin{itemize}
	\item \textbf{Analizador de variables:} Indica las variables que fueron configurado el tabata: Tiempo de descanso, tiempo de trabajo y la cantidad de series.
	\item  \textbf{Resultados generales:} Muestra los resultados del volumen de repeticiones  y el tiempo total empleado en la rutina de tabata.
	\item  \textbf{Endurance:} Vector de informaci\'on que permite construir la gr\'afica de la rutina tabata, a partir de los siguientes par\'ametros:
	   \begin{itemize}  
   	\item \textbf{uid:} C\'odigo \'unico de identificaci\'on de la gr\'afica.
   	\item \textbf{label:} Nombre de la gr\'afica.
   	\item \textbf{showLine:} Condici\'on que permite dibujar la linea de la tendencia de la gr\'afica.
    \item \textbf{data:} Vector de datos que conforma la gr\'afica, en donde los datos del eje x, representa el tiempo de rutina (en segundos) y los datos del eje y, significa las repeticiones acumulada durante ese tiempo de rutina.
   \end{itemize}     
    \item \textbf{Potencia:} Cantidad de repeticiones m\'axima del movimiento, en el menor tiempo posible.
    \item \textbf{Velocidad:} Se divide en dos resultados:
           \begin{itemize}
       \item Cantidad promedio de repeticiones en una serie.
       \item Tiempo promedio por una repetici\'on
       \end{itemize}
\end{itemize} 
\subsubsection{Web}\mbox{} \\
\subsubsection{Consola (Windows)}\mbox{} \\

\subsection{Herramienta para el an\'alisis de datos} \label{ins:toolsAn}
Se utiliz\'o el software, Microsoft Excel, para la tabulaci\'on y organizaci\'on de los datos.