\section{Dise\~no de la metodolog\`ia}\label{dis}
Esta secci\'on se presenta todos los dise\~nos y c\'alculos matem\'aticos y estad\'isticos encontrados en los resultados. 
\subsection{Asignaci\'on de valores de etiquetas y rangos de identificaci\'on}\label{dis:asig}
Estos valores permite identificar el paso de cada movimiento a partir de las siguientes variables:
\begin{itemize}
\item \textbf{Offset:} Valor de distribuci\'on de etiquetas a cada paso del movimiento sin tomar el paso inicial (Debido que la etiqueta del paso inicial siempre tendr\'a un valor de cero).
\begin{formula}[H]
	\centering
	\caption{Offset de etiquetas}
	\label{frm:offsetEt}
	\begin{equation}
offset = \frac{1}{pasos-1}
	\end{equation}
	\textbf{Fuente:} Propuesto por el autor de tesis
\end{formula}
\item \textbf{Vector de etiquetas:} A cada paso se le distribuye un valor \'unico, cuyo valor se calcula a partir del offset de la etiqueta anterior.
\begin{formula}[H]
	\centering
	\caption{Asignaci\'on de etiquetas}
	\label{frm:vecEtiq}
	\begin{equation}
\begin{matrix}
etiquetas=[0, etiqueta(1), ..., etiqueta(paso), ..., 1],\; donde
\\
\\
etiqueta(paso) =
\left\{\begin{matrix}
0 & si\; es\; el\; primer \; paso
\\
etiqueta(paso-1)+offset & si\; es\; un\; paso\; intermedio
\\ 
1 & si\; es\; el\; ultimo\; paso
\end{matrix}\right.
\end{matrix}
	\end{equation}
	\textbf{Fuente:} Propuesto por el autor de tesis
\end{formula} 

\item \textbf{Valor de identificaci\'on:} N\'umero que representa la distribuci\'on de pasos por partes iguales:
\begin{formula}[H]
	\centering
	\caption{Valor de identificaci\'on de pasos}
	\label{frm:idenStep}
	\begin{equation}
valor \: de \: identificacion = \frac{1}{pasos}
	\end{equation}
	\textbf{Fuente:} Propuesto por el autor de tesis
\end{formula} 

\item \textbf{Rango de identificaci\'on:} rango m\'aximo n\'umerico c\'alculado a partir del valor de identificaci\'on.
\begin{formula}[H]
	\centering
	\caption{Rango m\'aximo de identificaci\'on de un paso}
	\label{frm:idenStep}
	\begin{equation}
\begin{matrix}
rango = [[inferior,superior]] \\
\\
rango(paso)=\left\{\begin{matrix}
inferior(paso)= \left\{\begin{matrix}
0 & paso\: inicial\\ 
superior(paso-1) & paso\: no\: inicial
\end{matrix}\right.\\ 
\\
superior(paso)= \left\{\begin{matrix}
inferior(paso)+identificacion & paso\, no\, final\\ 
1 & paso\: final

\end{matrix}\right.\\ 
\end{matrix}\right.
\end{matrix}
	\end{equation}
	\textbf{Fuente:} Propuesto por el autor de tesis
\end{formula} 
\end{itemize}
\subsection{C\'alculo indirecto de la altura del usuario} \label{dis:height}
Para el presente proyecto se midi\'o la altura del usuario, a partir de la diferencia entre la altura de la cabeza y la altura promedio de lo p\'ies.
	\begin{formula}[H]
	\centering
	\caption{Altura del usuario}
	\label{frm:alturaUser}
	\begin{equation}
y_{usuario}=y_{cabeza}-\frac{y_{pieDerecho}+y_{pieIzquierdo}}{2}
	\end{equation}
			\textbf{Fuente:} Planteado por el autor de tesis
\end{formula} 
\subsection{C\'alculo de distancia de profundidad m\'inimma y m\'axima} \label{dis:deep}
Para los siguientes c\'alculos se utiliz\'o la hoja de observaciones de profundidad (Ver anexo, tabla  \ref{tab:obsDepth}), aplicando la siguientes funciones de Microsoft excel (Versi\'on ingl\'es):
\begin{itemize}
       \item \textbf{AVERAGE}: funci\'on para determinar la altura promedio.
\begin{formula}[H]
	\centering
	\caption{c\'alculo de altura promedio}
	\label{frm:avgHeight}
	\begin{equation}
	Altura \: promedio =AVERAGE([Altura])
	\end{equation}
		\textbf{Fuente:} Elaborado por el autor a partir de funci\'on de excel
\end{formula}
       \item \textbf{STDEV}: funci\'on para determinar la desviaci\'on est\'andar de la altura.
\begin{formula}[H]
	\centering
	\caption{c\'alculo de desviaci\'on est\'andar de la altura}
	\label{frm:stdevHeight}
	\begin{equation}
	desviacion \:  de \: altura =STDEV([Altura])
	\end{equation}
		\textbf{Fuente:} Elaborado por el autor a partir de funci\'on de excel
\end{formula}
       \item \textbf{MAX}: funci\'on para determinar la profundidad m\'axima entre el kinect y el atleta.
       \begin{formula}[H]
	\centering
	\caption{c\'alculo de la profundidad m\'axima}
	\label{frm:maxDepth}
	\begin{equation}
	Profundidad \: Max =MAX([Profundidad])
	\end{equation}
		\textbf{Fuente:} Elaborado por el autor a partir de funci\'on de excel
\end{formula}
       \item \textbf{MIN}: funci\'on para determinar la profundidad m\'inima entre el kinect y el atleta.
              \begin{formula}[H]
	\centering
	\caption{c\'alculo de la profundidad m\'inima}
	\label{frm:minDepth}
	\begin{equation}
	Profundidad \:  Min =MIN([Profundidad])
	\end{equation}
		\textbf{Fuente:} Elaborado por el autor a partir de funci\'on de excel.
\end{formula}
\end{itemize}
\subsection{Eventos del Kinect} \label{dis:even}
De acuerdo a las hoja de observaciones de extraci\'on de datos de v\'ideos (Ver tabla \ref{tab:obsVideoData}), se recupera las siguiente informaci\'on:
\begin{itemize}
\item \textbf{Eventos}: conjunto de datos del seguimiento de esqueleto, durante un intervalo de tiempo.
\begin{formula}[H]
	\centering
	\caption{matriz de eventos del Kinect}
	\label{frm:event}
	\begin{equation}
\begin{matrix}
Esqueleto & [SkeletonId, Joint, status, x, y, z] \\ 
Evento & [EventoIndex, TotalTime, esqueleto]  \\
\\ 
Eventos & 
\left.\begin{matrix}
Tiempo \: inicial\\ 
\\ 
\\
Tiempo \: final
\end{matrix}\right\}
\begin{matrix}
\left.\begin{matrix}
Evento \: inicial 
\end{matrix}\right\}\\ 
\left.\begin{matrix}
.. \\ 
Eventos \: no \: inicial \\ 
 ..
\end{matrix}\right\}
\end{matrix}
\begin{bmatrix}
Evento_{0}\\ 
Evento_{1}\\ 
Evento_{2}\\ 
Evento_{x}
\end{bmatrix}
\end{matrix}
	\end{equation}
		\textbf{Fuente:} Propuesto por el autor de tesis.
\end{formula}
\item \textbf{Tiempo relativo}: Describe el tiempo de la repetici\'on, cuyo valor significa la diferencia entre el tiempo total de un evento no inicial y el tiempo total del primer evento.
              \begin{formula}[H]
	\centering
	\caption{c\'alculo del tiempo de la repetici\'on}
	\label{frm:relativeTime}
	\begin{equation}
	relative \: time = TotalTime_{Evento\: x}-TotalTime_{Evento\: inicial}
	\end{equation}
		\textbf{Fuente:} Propuesto por el autor de tesis.
\end{formula}
\item \textbf{Distancia euclediana}: Describe el desplazamiento de una articulaci\'on, bas\'andose en la posici\'on del primer evento inicial y la posici\'on de un evento no inicial:
\begin{formula}[H]
	\centering
	\caption{desplazamiento de una articulaci\'on}
	\label{frm:desplazaUser}
	\begin{equation}
|r|=\sqrt{(x_{joint}-x_{o})^{2}+(y_{joint}-y_{o})^{2}+(z_{joint}-z_{o})^{2}}
	\end{equation}
	\textbf{Fuente:} Distancia euclediana \cite[p.~423]{ayres2001calculo}
\end{formula}  
\end{itemize}
As\'i mismo, con la herramienta de Excel se realiz\'o un an\'alisis de regresi\'on  de los movimiento de animaci\'on y tenis de mesa, tomando en cuenta  la distancia recorrida de la mun\~eca -i.e. Variable independiente- y su respectivo tiempo -i.e. Variable dependiente-, a partir de una muestra de 10 a 15 repeticiones, con la finalidad de comparar los coeficientes de determinaci\'on de las siguientes lineas de tendencias:
\begin{itemize}
\item Exponencial
\item Lineal
\item Logar\'itmica
\item Polinomial de grado 2
\item Polinomial de grado 3
\item Polinomial de grado 4
\item Potencia
\end{itemize}
\subsection{Captura de datos durante una rutina}\label{dis:recognitionMove}
Durante la rutina, los datos del seguimiento de esqueleto son recuperado a partir del kit de desarrollo de software del kinect, dichos datos son almacenados en la siguiente estructura:
\begin{itemize}
\item \textbf{i:} Valor n\'umerico \'unico que identifica una articulaci\'on.
\item \textbf{x:} Posici\'on horizontal de la articulaci\'on, dibujado en una imagen de dos dimensiones.
\item \textbf{y:} Posici\'on vertical de la articulaci\'on, dibujado en una imagen de dos dimensiones.
\item \textbf{fm:} Factor del movimiento proporcionado por la base de datos de gesturas.
\item \textbf{p:} Valor n\'umerico que identifica el paso que esta realizando un atleta.
\item \textbf{tiempo:} Tiempo de captura de datos.
\item \textbf{joint:} Vector que almacena la posici\'on de una articulaci\'on en una figura de dos dimensiones.
\item \textbf{Esqueleto:} Vector que almacena todos los vectores de articulaciones del esqueleto humano.
\item \textbf{step:} Vector que almacena informaci\'on del seguimiento esqueleto, factor del movimiento y el tiempo de un paso.
\item \textbf{repetici\'on:} Vector que almacena la informaci\'on de cada de paso de un movimiento.
\item \textbf{serie:} Vector que almacena las repeticiones del movimiento durante una serie.
\item \textbf{series:} Vector que almacena la informaci\'on de cada serie de la rutina.
\end{itemize}
\begin{formula}[H]
	\centering
	\caption{Matriz de datos capturados durante una rutina}
	\label{frm:MatrizDatosRepeticion}
	\begin{equation}
\begin{matrix}
i & enumerador\: de\: la\: articulacion \\ 
x & distancia\, horizontal \: (pixeles) \\ 
y & distancia\, vertical\: (pixeles) \\ 
joint_{i}& [i,x,y] \\ 
 & \\ 
esqueleto & \begin{bmatrix}
joint_{0} \\ 
... \\ 
joint_{i}\\ 
... \\ 
joint_{24}
\end{bmatrix}  \\ 
 & \\ 
fm & factor \, del \, movimiento \\ 
p  & paso \, del \,movimiento \\ 
t  & tiempo \, total \\ 
step_{p}  & [fm,p,esqueleto, tiempo] \\
 & \\ 
Repeticion & [step_{0}, ...,  step_{col}, ..., step_{last}] \\
serie & [repeticion] \\
series & [serie]
\end{matrix}
	\end{equation}
	\textbf{Fuente:} Propuesto por el autor de tesis
\end{formula}
De igual manera se presenta el algoritmo de captura de datos de la rutina, dicho algoritmo se ejecuta durante el tiempo de trabajo de tabata, en donde comienza con una subrutina para obtener el valor de factor de movimiento en tiempo real, posteriormente ese valor se compara a partir de los intervalos de confianza, si el valor no se encuentra en ning\'un intervalo, se debe obtener un nuevo factor de movimiento (Durante un per\'iodo de 0.033 segundos). En caso contrario se obtiene el valor de la etiqueta del paso. Luego se revisa si hay un dato registrado en relaci\'on al paso actual, en caso que haya un registro asociado, se debe eliminar los datos del paso siguiente (Esto ocurre, si el usuario se salta un paso del movimiento) y as\'i mismo se elimina el registro del paso actual, despu\'es se continua con el proceso que no haya ning\'un registro del paso, la cual se debe obtener toda la informaci\'on del seguimiento esqueleto y seguidamente se almacena el registro en el vector de repetici\'on. Finalmente se chequea la cantidad de registro del  vector de repetici\'on, en caso que falte un registro se debe reiniciar nuevamente el algoritmo, por lo contrario, se almacena en el registro de repetici\'on.
\begin{figure}[H]
	\caption{Algoritmo de captura de datos durante la rutina}
	\label{fig:capturaDatos}
	\centering
	\includegraphics[width=430px,height=470px]{graphics/algoritmoDeteccion.png} \\
	\textbf{Fuente:} Elaborado por el autor de tesis
\end{figure}
\subsection{Validaci\'on del modelo de reconocimiento de movimiento}\label{dis:validate}
El presente proyecto se utiliz\'o dos m\'etodos de validaciones cruzadas \cite{perez2015analisis}:
\begin{itemize}
\item \textbf{Hold out:} La muestra de datos se separa en dos conjunto, uno para construir y entrenar el modelo -i.e. Build- y otro para realizar pruebas que dan validez al error del modelo.
\item \textbf{K-fold:} La muestra se divide en k subconjuntos, y en cada subconjunto se aplica el m\'etodo Hold-out.
\end{itemize}  
\begin{table}[H]
\begin{center}
\caption{Validaci\'on cruzada, 3-fold,  Tenis de mesa}
\label{tab:KfoldTenis}
\begin{tabular}{|l|l|l|}
\hline
\multicolumn{3}{|c|}{\textbf{Muestra de v\'ideos}} \\ \hline
\multicolumn{3}{|c|}{1, 2, 3, 4, 5, 6} \\ \hline
\textbf{Modelo} & \textbf{Builds} & \textbf{Test} \\ \hline
1 & 1, 2, 3, 4, 5 & 6 \\ \hline
2 & 1, 2, 4, 5, 6 & 3 \\ \hline
3 & 2, 3, 4, 5, 6 & 1 \\ \hline
\multicolumn{3}{l}{\textbf{Fuente:} Elaborado por el autor de tesis}
\end{tabular}
\end{center}
\end{table}
\begin{table}[H]
\begin{center}
\caption{Validaci\'on cruzada, 3-fold,  Animaci\'on}
\label{tab:KfoldAnimacion}
\begin{tabular}{|l|l|l|}
\hline
\multicolumn{3}{|c|}{\textbf{Muestra de v\'ideos}} \\ \hline
\multicolumn{3}{|c|}{1, 2, 3, 4, 5, 6, 7} \\ \hline
\textbf{Modelo} & \textbf{Builds} & \textbf{Test} \\ \hline
1 & 1, 2, 3, 4, 5, 6 & 1 \\ \hline
2 & 1, 2, 3, 5, 6, 7 & 4 \\ \hline
3 & 1, 2, 3, 4, 5, 6 & 7 \\ \hline
\multicolumn{3}{l}{\textbf{Fuente:} Elaborado por el autor de tesis}
\end{tabular}
\end{center}
\end{table}
\begin{table}[H]
\begin{center}
\caption{Validaci\'on cruzada, 3-fold,  Taekwondo}
\label{tab:KfoldTaekwondo}
\begin{tabular}{|l|l|l|}
\hline
\multicolumn{3}{|c|}{\textbf{Muestra de videos}} \\ \hline
\multicolumn{3}{|c|}{1, 2, 3, 4, 5, 6, 7, 8, 9, 10, 11, 12, 13, 14, 15, 16} \\ \hline
\textbf{Modelo} & \textbf{Builds} & \textbf{Test} \\ \hline
1 & 1, 2, 3, 4, 5, 6, 7, 8, 9, 11, 12, 13, 14, 15, 16 & 10 \\ \hline
2 & 1, 2, 3, 4, 5, 6, 7, 8, 9, 10, 12, 13, 14, 15, 16 & 11 \\ \hline
3 & 1, 2, 3, 4, 5, 6, 7, 8, 9, 10, 11, 13, 14, 15, 16 & 12 \\ \hline
\multicolumn{3}{l}{\textbf{Fuente:} Elaborado por el autor de tesis}
\end{tabular}
\end{center}
\end{table}
Posteriormente en la etapa de pruebas, se realiz\'o  por cada modelo un hoja de observaciones de valores reales y pron\'onosticados (ver anexo, tabla \ref{tab:obsErrores}), de modo de encontrar los siguientes errores de pron\'osticos \cite{erroresPronostico}:
\begin{itemize}
\item \textbf{Error Medio Pron\'osticado (\acrshort{EMP}):} Promedio de diferencia entre el valor real y pron\'osticado, la cual puede tener tres interpretaciones distintas:
	\begin{itemize}
	\item \textbf{Valor positivo:} En promedio los valores pron\'osticado est\'an por arriba de los valores reales.
	\item \textbf{Valor negativo:} En promedio los valores pron\'osticado est\'an por debajo de los valores reales.
	\item \textbf{Valor cero:} Los valores pron\'osticados pueden estar por debajo o por arriba de los valores reales.
	\end{itemize}
\begin{formula}[H]
	\centering
	\caption{C\'alculo del error medio pron\'osticado}
	\label{frm:empMath}
	\begin{equation}
EMP=\frac{\sum_{0}^{n}(Real_{x}-Pronostic_{x}\cdot10^{-6})}{n}
	\end{equation}
	\textbf{Fuente:} Formula adaptada al proyecto, a partir de la f\'ormula EMP \cite{videoErrores}
\end{formula}  
\item \textbf{Desviaci\'on Media absoluta (\acrshort{DMA}):} Promedio de diferencia absoluta entre el valor real y pron\'osticado, que muestra la dispersi\'on con respecto al valor real.
 \begin{formula}[H]
	\centering
	\caption{C\'alculo de la desviaci\'on media absoluta}
	\label{frm:MADMath}
	\begin{equation}
DMA=\frac{\sum_{0}^{n}( \left |  Real_{x}-Pronostic_{x}\cdot10^{-6} \right |)}{n}
	\end{equation}
	\textbf{Fuente:} Formula adaptada al proyecto, a partir de la f\'ormula DMA \cite{videoErrores}
\end{formula}  
\item \textbf{Ra\'iz del error cuadr\'atico medio (RECM):} Es la desviaci\'on est\'andar de los errores de predicci\'on, la cual indica qu\'e tan disperso se encuentra los errores con respecto al error medio pron\'osticado.
 \begin{formula}[H]
	\centering
	\caption{C\'alculo de la Ra\'iz del error cuadr\'atico medio}
	\label{frm:RECMMath}
	\begin{equation}
RECM=\sqrt{\frac{\sum_{0}^{n}((Real-Pronostic\cdot10^{-6})-EMP)^{2}}{n}}
	\end{equation}
	\textbf{Fuente:} Formula adaptada al proyecto, a partir de la f\'ormula RECM \cite{GEORCMETUT}
\end{formula}  
\end{itemize}
Luego de encontrar los errores, se debe seleccionar el modelo que tenga el menor error \acrshort{RECM}, y posteriormente encontrar los errores de la muestra total:
 \begin{formula}[H]
	\centering
	\caption{EMP de la muestra total}
	\label{frm:EmpAll}
	\begin{equation}
EMP_{muestra\, total}=AVERAGE([EMP_{modelo}])
	\end{equation}
	\textbf{Fuente:} Propuesto por el autor de tesis, en base a las f\'ormulas de excel (versi\'on en ingl\'es)
\end{formula}  
 \begin{formula}[H]
	\centering
	\caption{MAD de la muestra total}
	\label{frm:MadAll}
	\begin{equation}
MAD_{muestra\, total}=AVERAGE([MAD_{modelo}])
	\end{equation}
	\textbf{Fuente:} Propuesto por el autor de tesis, en base a las f\'ormulas de excel (versi\'on en ingl\'es)
\end{formula}  
	 \begin{formula}[H]
	\centering
	\caption{RECM de la muestra total}
	\label{frm:RecmAll}
	\begin{equation}
RECM_{muestra\, total}=AVERAGE([RECM_{modelo}])
	\end{equation}
	\textbf{Fuente:} Propuesto por el autor de tesis, en base a las f\'ormulas de excel (versi\'on en ingl\'es)
\end{formula} 
Los errores de la muestra total ayuda aceptar o rechazar el modelo; si el el error \acrshort{RECM} es menor al valor de identificaci\'on (ver f\'ormula \ref{frm:idenStep}), se aprueba. En caso que sea mayor o igual, se rechaza, debido que existe secciones que no podr\'a identificar el paso, tal como se muestra en la siguiente figura:
\begin{figure}[H]
	\caption{Aprobaci\'on o rechazo del modelo}
	\label{fig:AproveOrDennie}
	\centering
	\includegraphics[width=430px,height=180px]{graphics/aceptacionModelo.png} \\
	\textbf{Fuente:} Elaborado por el autor de tesis
\end{figure}
Finalmente, por cada modelo aprobado se debe encontrar el factor de reconocimiento y sus respectivo rango de confiabilidad:
\begin{formula}[H]
	\centering
	\caption{Factor de reconocimiento e intervalos de confianza}
	\label{frm:rangoConfiabilidad}
	\begin{equation}
\begin{matrix}
recognition=\frac{RECM_{muestra\, total}}{valor\: de\: identificacion}  \\ 
\\
intervalo\: de\: confianza = ic =[[inferior,superior]] \\
\\
ic(paso)=\left\{\begin{matrix}
inferior(paso)=\left\{\begin{matrix}
0 & si\, es\, el\, primer\, paso\\ \\ 
etiqueta(paso)-\frac{recognition*valor\, de\, identificacion}{2} & si\: es\: paso\: intermedio \\ 
\\
1-(recognition*valor\, de\, identificacion)& si\, es\, el\, ultimo\, paso
\end{matrix}\right. \\ 
\\ 
inferior(paso)=\left\{\begin{matrix}
recognition*valor\, de\, identificacion& si\, es\, el\, primer\, paso \\
\\
etiqueta(paso)+\frac{recognition*valor\, de\, identificacion}{2} & si\: es\: paso\: intermedio \\ \\ 
1 & si\: es\: el\: ultimo\: paso
\end{matrix}\right.
\end{matrix}\right.
\end{matrix}
	\end{equation}
	\textbf{Fuente:} Propuesto por el autor de tesis
\end{formula} 
\subsection{Dise\~no de los resultados de una rutina tabata} \label{dis:results}
En la presente secci\'on se determina los resultados de la rutina de tabata, dichos resultado son encontrados a partir del seguimiento del esqueleto, la cual proporciona informaci\'on de las variables del detalle de paso y repetici\'on (Ver f\'ormula \ref{frm:MatrizDatosRepeticion}), con el fin objetivo de encontrar la siguiente informaci\'on:
\begin{itemize}
\item \textbf{Volumen de repeticiones:} Sumatoria de las cantidades totales de repeticiones de una serie.  
\begin{code}[H]
	\caption{funci\'on para obtener las repeticiones totales de una rutina}
	\label{code:getRepetitions}
	\begin{lstlisting}
public int getRepetitions()
{
	int repetitions = 0;
	foreach(serie _serie in Series)
	{
		repetitions += _serie.length;
	}
	return repetitions;
}
	\end{lstlisting}
	\textbf{Fuente:} Elaborado por el autor en sintaxis de c\#.
\end{code}

\item \textbf{Duraci\'on:} Tiempo total que se emplea en una rutina, tomando en cuenta todas las series de trabajo y descanso.
\begin{formula}[H]
	\centering
	\caption{c\'alculo de la duraci\'on de tiempo de una rutina}
	\label{eq:DurationTime}
	\begin{equation}
	Duration = \sum_{i=0}^{series}restTime +\sum_{i=0}^{series}workTime = series(restTime+workTime)
	\end{equation}
		\textbf{Fuente:} Elaborado por el autor de tesis
\end{formula}
\item \textbf{Resistencia:} Construye la estructura de la gr\'afica de trabajo de la rutina a partir del recorrido de todas la series, posteriormente en cada serie se recorre todas las repeticiones y luego en cada repetici\'on se almacena el tiempo acumulado de la rutina y la cantidad de repeticiones que lleva el atleta a ese momento.
\begin{code}[H]
	\caption{funci\'on para obtener los resultados del endurance}
	\label{code:getEndurance}
	\begin{lstlisting}
public endurance getEndurance()
{
	List<endurance> _endurance = new List<endurance>();
	for(int i = 0; i < series.length; i++)
	{
		List<data> _data = new List<data>();
		
		serie _serie = series[i];
		for(int j = 1; j <= _serie.length; j++)
		{
			Repetition repetition = _serie[j-1];
			int lastStep = repetition.length-1;
			double tiempoAcumulado = repetition[lastStep].tiempo;

			_data.add(new data(){
				x = tiempoAcumulado,
				y = j
			});
		}
		
		_endurance.add(new endurance(){
			uid      = $"s{i+1}",
			label    = $"Serie {i+1}",
			showLine = true,
			data     =  _data
		});
	}
	return _endurance;
}
	\end{lstlisting}
	\textbf{Fuente:} Elaborado por el autor en sintaxis de c\#.
\end{code} 

\item \textbf{Potencia:} Resultado que muestra la cantidad m\'axima de repeticiones en el menor tiempo posible, la cual se encuentra a partir del recorrido de las series, luego en cada serie se determina la cantidad de repeticiones totales y el tiempo acumulado de la \'ultima repetici\'on, encontrando: 
	\begin{itemize}
	\item Una mayor cantidad de repeticiones almacenada anteriormente
	\item La misma cantidad repeticiones almacenada anteriormente, pero se verifica si el tiempo acumulado es menor.
	\end{itemize}
\begin{code}[H]
	\caption{funci\'on para obtener la potencia}
	\label{code:getEndurance}
	\begin{lstlisting}
public power getPower()
{
	power _power = new power(){repetition = 0, time = 0};
	foreach(serie _serie in series)
	{
		Repetition lastRepetition = _serie[_serie.length-1];
		int lastStep = lastRepetition.length-1;
		double tiempoAcumulado = lastRepetition[lastStep].tiempo;
		
		if (_serie.length > _power.repetition)
		{
			_power.repetition = _serie.length;
			_power.time = tiempoAcumulado;
		} 
		else if (serie.length == _power.repetition)
		{
			if(tiempoAcumulado <  _power.time)
			{
				_power.repetition = _serie.length;
				_power.time = tiempoAcumulado;
			}
		}
	}
	return _power;
}
	\end{lstlisting}
	\textbf{Fuente:} Elaborado por el autor en sintaxis de c\#.
\end{code} 

\item \textbf{Velocidad}: Raz\'on de cambio separado por:
	\begin{itemize}
	\item \textbf{Repeticiones por serie:} Variable que se determina a partir del promedio total de repeticiones por serie.
		\item \textbf{Tiempo por repeticion:} Variable que se determina a partir del promedio de la diferencia entre el tiempo del paso final y tiempo del paso inicial -i.e. Tiempo de una repetici\'on- de cada repetici\'on realizada por el atleta.
	\end{itemize}
\end{itemize}


\begin{code}[H]
	\caption{funci\'on para obtener las velocidades de las rutinas}
	\label{code:getTimeOfRepetitions}
	\begin{lstlisting}
public speed getSpeed()
{
	speed _speed = new speed();
	List<double> timeOfRepetitions = new List<double>();
	List<int> repetitionsOfSerie = new List<double>();
	
	foreach(serie _serie in series)
	{
		repetitionsOfSerie.add(_serie.length);
		foreach(Repetition repetition in _serie)
		{
			int lastStep = repetition.length-1;
			double tiempoFinal = repetition[lastStep].tiempo;
			double tiempoInicial = repetition[0].tiempo;
			double time = tiempoFinal - tiempoInicial;
			timeOfRepetitions.add(time);
		}			
	}
	_speed.repetitionBySerie = Convert.toInt32(
		 repetitionsOfSerie.Average());
	_speed.timeByRepetition = timeOfRepetitions.Average();
	return speed;
}
	\end{lstlisting}
	\textbf{Fuente:} Elaborado por el autor en sintaxis de c\#.
\end{code} 
