\section{DISE�O DE LA METODOLOG�A}
\subsection{C\'alculo de distancia de profundidad m\'inimma y m\'axima}
Para los siguientes c\'alculos se utiliz\'o la hoja de observaciones de profundidad (Ver anexo, tabla  \ref{tab:obsDepth}), aplicando la siguientes funciones de Microsoft excel (Versi\'on ingl\'es):
\begin{itemize}
       \item \textbf{AVERAGE}: funci\'on para determinar la altura promedio.
\begin{formula}[H]
	\centering
	\caption{c\'alculo de altura promedio}
	\label{frm:avgHeight}
	\begin{equation}
	Altura \: promedio =AVERAGE([Altura])
	\end{equation}
		\textbf{Fuente:} Elaborado por el autor a partir de funci\'on de excel
\end{formula}
       \item \textbf{STDEV}: funci\'on para determinar la desviaci\'on est\'andar de la altura.
\begin{formula}[H]
	\centering
	\caption{c\'alculo de desviaci\'on est\'andar de la altura}
	\label{frm:stdevHeight}
	\begin{equation}
	desviacion \:  de \: altura =STDEV([Altura])
	\end{equation}
		\textbf{Fuente:} Elaborado por el autor a partir de funci\'on de excel
\end{formula}
       \item \textbf{MAX}: funci\'on para determinar la profundidad m\'axima entre el kinect y el atleta.
       \begin{formula}[H]
	\centering
	\caption{c\'alculo de la profundidad m\'axima}
	\label{frm:maxDepth}
	\begin{equation}
	Profundidad \: Max =MAX([Profundidad])
	\end{equation}
		\textbf{Fuente:} Elaborado por el autor a partir de funci\'on de excel
\end{formula}
       \item \textbf{MIN}: funci\'on para determinar la profundidad m\'inima entre el kinect y el atleta.
              \begin{formula}[H]
	\centering
	\caption{c\'alculo de la profundidad m\'inima}
	\label{frm:minDepth}
	\begin{equation}
	Profundidad \:  Min =MIN([Profundidad])
	\end{equation}
		\textbf{Fuente:} Elaborado por el autor a partir de funci\'on de excel.
\end{formula}
\end{itemize}
\subsection{Dise\~no de los resultados de una rutina tabata}
En la presente secci\'on se determina los resultados de la rutina de tabata, dichos resultado son encontrados a partir del seguimiento del esqueleto, la cual proporciona informaci\'on de las variables del detalle de paso y repetici\'on (Ver variables independientes \ref{vr:7oi:indep}), con el fin objetivo de encontrar la siguiente informaci\'on:
\begin{itemize}
\item \textbf{Volumen de repeticiones:} Sumatoria de las cantidades totales de repeticiones de una serie.  
\begin{code}[H]
	\caption{funci\'on para obtener las repeticiones totales de una rutina}
	\label{code:getRepetitions}
	\begin{lstlisting}
public int getRepetitions()
{
	int repetitions = 0;
	foreach(serie _serie in Series)
	{
		repetitions += _serie.length;
	}
	return repetitions;
}
	\end{lstlisting}
	\textbf{Fuente:} Elaborado por el autor en sintaxis de c\#.
\end{code}

\item \textbf{Duraci\'on:} Tiempo total que se emplea en una rutina, tomando en cuenta todas las series de trabajo y descanso.
\begin{formula}[H]
	\centering
	\caption{c\'alculo de la duraci\'on de tiempo de una rutina}
	\label{eq:DurationTime}
	\begin{equation}
	Duration = \sum_{i=0}^{series}restTime +\sum_{i=0}^{series}workTime = series(restTime+workTime)
	\end{equation}
		\textbf{Fuente:} Elaborado por el autor de tesis
\end{formula}
\item \textbf{Resistencia:} Construye la estructura de la gr\'afica de trabajo de la rutina a partir del recorrido de todas la series, posteriormente en cada serie se recorre todas las repeticiones y luego en cada repetici\'on se almacena el tiempo acumulado de la rutina y la cantidad de repeticiones que lleva el atleta a ese momento.
\begin{code}[H]
	\caption{funci\'on para obtener los resultados del endurance}
	\label{code:getEndurance}
	\begin{lstlisting}
public endurance getEndurance()
{
	List<endurance> _endurance = new List<endurance>();
	for(int i = 0; i < series.length; i++)
	{
		List<data> _data = new List<data>();
		
		serie _serie = series[i];
		for(int j = 1; j <= _serie.length; j++)
		{
			Repetition repetition = _serie[j-1];
			int lastStep = repetition.length-1;
			double tiempoAcumulado = repetition[lastStep].tiempo;

			_data.add(new data(){
				x = tiempoAcumulado,
				y = j
			});
		}
		
		_endurance.add(new endurance(){
			uid      = $"s{i+1}",
			label    = $"Serie {i+1}",
			showLine = true,
			data     =  _data
		});
	}
	return _endurance;
}
	\end{lstlisting}
	\textbf{Fuente:} Elaborado por el autor en sintaxis de c\#.
\end{code} 

\item \textbf{Potencia:} Resultado que muestra la cantidad m\'axima de repeticiones en el menor tiempo posible, la cual se encuentra a partir del recorrido de las series, luego en cada serie se determina la cantidad de repeticiones totales y el tiempo acumulado de la \'ultima repetici\'on, encontrando: 
	\begin{itemize}
	\item Una mayor cantidad de repeticiones almacenada anteriormente
	\item La misma cantidad repeticiones almacenada anteriormente, pero se verifica si el tiempo acumulado es menor.
	\end{itemize}
\begin{code}[H]
	\caption{funci\'on para obtener la potencia}
	\label{code:getEndurance}
	\begin{lstlisting}
public power getPower()
{
	power _power = new power(){repetition = 0, time = 0};
	foreach(serie _serie in series)
	{
		Repetition lastRepetition = _serie[_serie.length-1];
		int lastStep = lastRepetition.length-1;
		double tiempoAcumulado = lastRepetition[lastStep].tiempo;
		
		if (_serie.length > _power.repetition)
		{
			_power.repetition = _serie.length;
			_power.time = tiempoAcumulado;
		} 
		else if (serie.length == _power.repetition)
		{
			if(tiempoAcumulado <  _power.time)
			{
				_power.repetition = _serie.length;
				_power.time = tiempoAcumulado;
			}
		}
	}
	return _power;
}
	\end{lstlisting}
	\textbf{Fuente:} Elaborado por el autor en sintaxis de c\#.
\end{code} 

\item \textbf{Velocidad}: Raz\'on de cambio separado por:
	\begin{itemize}
	\item \textbf{Repeticiones por serie:} Variable que se determina a partir del promedio total de repeticiones por serie.
		\item \textbf{Tiempo por repeticion:} Variable que se determina a partir del promedio de la diferencia entre el tiempo del paso final y tiempo del paso inicial -i.e. Tiempo de una repetici\'on- de cada repetici\'on realizada por el atleta.
	\end{itemize}
\end{itemize}


\begin{code}[H]
	\caption{funci\'on para obtener las velocidades de las rutinas}
	\label{code:getTimeOfRepetitions}
	\begin{lstlisting}
public speed getSpeed()
{
	speed _speed = new speed();
	List<double> timeOfRepetitions = new List<double>();
	List<int> repetitionsOfSerie = new List<double>();
	
	foreach(serie _serie in series)
	{
		repetitionsOfSerie.add(_serie.length);
		foreach(Repetition repetition in _serie)
		{
			int lastStep = repetition.length-1;
			double tiempoFinal = repetition[lastStep].tiempo;
			double tiempoInicial = repetition[0].tiempo;
			double time = tiempoFinal - tiempoInicial;
			timeOfRepetitions.add(time);
		}			
	}
	_speed.repetitionBySerie = Convert.toInt32(
		 repetitionsOfSerie.Average());
	_speed.timeByRepetition = timeOfRepetitions.Average();
	return speed;
}
	\end{lstlisting}
	\textbf{Fuente:} Elaborado por el autor en sintaxis de c\#.
\end{code} 