\section{Dise\~no de la metodolog\'ia}\label{dis}
Esta secci\'on se presenta todos los dise\~nos y c\'alculos matem\'aticos y estad\'isticos encontrados en los resultados. 
\subsection{Asignaci\'on de valores de etiquetas y rangos de identificaci\'on}\label{dis:asig}
Estos valores permiten identificar el paso de cada movimiento a partir de las siguientes variables:
\begin{itemize}
\item \textbf{Offset:} Valor de distribuci\'on de etiquetas a cada paso del movimiento, sin tomar el paso inicial (Debido que la etiqueta del paso inicial siempre tendr\'a un valor de cero).
\begin{formula}[H]
	\centering
	\caption{Offset de etiquetas}
	\label{frm:offsetEt}
	\begin{equation}
offset = \frac{1}{pasos-1}
	\end{equation}
	\textbf{Fuente:} Propuesto por el autor de tesis
\end{formula}
\item \textbf{Vector de etiquetas:} A cada paso se le distribuye un valor \'unico, cuyo valor se calcula a partir del offset de la etiqueta anterior.
\begin{formula}[H]
	\centering
	\caption{Asignaci\'on de etiquetas}
	\label{frm:vecEtiq}
	\begin{equation}
\begin{matrix}
etiquetas=[0, etiqueta(1), ..., etiqueta(paso), ..., 1],\; donde
\\
\\
etiqueta(paso) =
\left\{\begin{matrix}
0 & Si\; es\; el\; primer \; paso
\\
etiqueta(paso-1)+offset & Si\; es\; un\; paso\; intermedio
\\ 
1 & Si\; es\; el\; \acute{u}ltimo\; paso
\end{matrix}\right.
\end{matrix}
	\end{equation}
	\textbf{Fuente:} Propuesto por el autor de tesis
\end{formula} 

\item \textbf{Valor de identificaci\'on:} N\'umero que representa la distribuci\'on de reconocimiento  de pasos por partes iguales:
\begin{formula}[H]
	\centering
	\caption{Valor de identificaci\'on de pasos}
	\label{frm:idenStep}
	\begin{equation}
valor \: de \: identificaci\acute{o}n = \frac{1}{pasos}
	\end{equation}
	\textbf{Fuente:} Propuesto por el autor de tesis
\end{formula} 

\item \textbf{Rango de identificaci\'on:} Rango m\'aximo num\'erico para identificar un paso.
\begin{formula}[H]
	\centering
	\caption{Rango m\'aximo de identificaci\'on de un paso}
	\label{frm:idenStep}
	\begin{equation}
\begin{matrix}
rango = [[inferior,superior]] \\
\\
rango(paso)=\left\{\begin{matrix}
inferior(paso)= \left\{\begin{matrix}
0 & paso\: inicial\\ 
superior(paso-1) & paso\: no\: inicial
\end{matrix}\right.\\ 
\\
superior(paso)= \left\{\begin{matrix}
inferior(paso)+identificaci\acute{o}n & paso\, no\, final\\ 
1 & paso\: final

\end{matrix}\right.\\ 
\end{matrix}\right.
\end{matrix}
	\end{equation}
	\textbf{Fuente:} Propuesto por el autor de tesis
\end{formula} 
\end{itemize}
\subsection{C\'alculo indirecto de la altura del usuario} \label{dis:height}
Para el presente proyecto se midi\'o la altura del usuario, a partir de la diferencia entre la altura de la cabeza y la altura promedio de los pies.
	\begin{formula}[H]
	\centering
	\caption{Altura del usuario}
	\label{frm:alturaUser}
	\begin{equation}
y_{usuario}=y_{cabeza}-\frac{y_{pie \: derecho}+y_{pie \: izquierdo}}{2}
	\end{equation}
			\textbf{Fuente:} Planteado por el autor de tesis
\end{formula} 
\subsection{C\'alculo de distancia de profundidad m\'inima y m\'axima} \label{dis:deep}
Para los siguientes c\'alculos se utilizaron la hoja de observaciones de profundidad (Ver anexo, tabla  \ref{tab:obsDepth}), aplicando las siguientes funciones de microsoft excel (versi\'on ingl\'es):
\begin{itemize}
       \item \textbf{Average}: Funci\'on para determinar la altura promedio.
\begin{formula}[H]
	\centering
	\caption{C\'alculo de altura promedio}
	\label{frm:avgHeight}
	\begin{equation}
	Altura \: promedio =AVERAGE([Altura])
	\end{equation}
		\textbf{Fuente:} Elaborado por el autor, a partir de funci\'on de excel
\end{formula}
       \item \textbf{Stdev}: Funci\'on para determinar la desviaci\'on est\'andar de la altura.
\begin{formula}[H]
	\centering
	\caption{C\'alculo de desviaci\'on est\'andar de la altura}
	\label{frm:stdevHeight}
	\begin{equation}
	desviaci\acute{o}n \:  de \: altura =STDEV([Altura])
	\end{equation}
		\textbf{Fuente:} Elaborado por el autor, a partir de funci\'on de excel
\end{formula}
       \item \textbf{Max}: Funci\'on para determinar la profundidad m\'axima entre el Kinect y el atleta.
       \begin{formula}[H]
	\centering
	\caption{C\'alculo de la profundidad m\'axima}
	\label{frm:maxDepth}
	\begin{equation}
	Profundidad \: Max =MAX([Profundidad])
	\end{equation}
		\textbf{Fuente:} Elaborado por el autor, a partir de funci\'on de excel
\end{formula}
       \item \textbf{Min}: Funci\'on para determinar la profundidad m\'inima entre el Kinect y el atleta.
              \begin{formula}[H]
	\centering
	\caption{C\'alculo de la profundidad m\'inima}
	\label{frm:minDepth}
	\begin{equation}
	Profundidad \:  Min =MIN([Profundidad])
	\end{equation}
		\textbf{Fuente:} Elaborado por el autor, a partir de funci\'on de excel.
\end{formula}
\end{itemize}
\subsection{Eventos del Kinect} \label{dis:even}
De acuerdo a la hoja de observaciones de extracci\'on de datos de v\'ideos (Ver tabla \ref{tab:obsVideoData}), se recupera la siguiente informaci\'on:
\begin{itemize}
\item \textbf{Eventos}: Conjunto de datos del seguimiento del esqueleto, durante un intervalo de tiempo.
\begin{formula}[H]
	\centering
	\caption{Matriz de eventos del Kinect}
	\label{frm:event}
	\begin{equation}
\begin{matrix}
Esqueleto & [SkeletonId, Joint, status, x, y, z] \\ 
Evento & [EventoIndex, TotalTime, esqueleto]  \\
\\ 
Eventos & 
\left.\begin{matrix}
Tiempo \: inicial\\ 
\\ 
\\
Tiempo \: final
\end{matrix}\right\}
\begin{matrix}
\left.\begin{matrix}
Evento \: inicial 
\end{matrix}\right\}\\ 
\left.\begin{matrix}
.. \\ 
Eventos \: no \: iniciales \\ 
 ..
\end{matrix}\right\}
\end{matrix}
\begin{bmatrix}
Evento_{0}\\ 
Evento_{1}\\ 
Evento_{2}\\ 
Evento_{x}
\end{bmatrix}
\end{matrix}
	\end{equation}
		\textbf{Fuente:} Propuesto por el autor de tesis.
\end{formula}
\item \textbf{Tiempo relativo}: Describe el tiempo de la repetici\'on, cuyo valor significa la diferencia entre el tiempo total de un evento no inicial y el tiempo total del primer evento.
              \begin{formula}[H]
	\centering
	\caption{C\'alculo del tiempo de la repetici\'on}
	\label{frm:relativeTime}
	\begin{equation}
	relative \: time = TotalTime_{Evento\: x}-TotalTime_{Evento\: inicial}
	\end{equation}
		\textbf{Fuente:} Propuesto por el autor de tesis.
\end{formula}
\item \textbf{Distancia euclidiana}: Describe el desplazamiento de una articulaci\'on, bas\'andose en la posici\'on del primer evento inicial y la posici\'on de un evento no inicial:
\begin{formula}[H]
	\centering
	\caption{Desplazamiento de una articulaci\'on}
	\label{frm:desplazaUser}
	\begin{equation}
|r|=\sqrt{(x_{joint}-x_{o})^{2}+(y_{joint}-y_{o})^{2}+(z_{joint}-z_{o})^{2}}
	\end{equation}
	\textbf{Fuente:} Distancia euclidiana \cite[p.~423]{ayres2001calculo}
\end{formula}  
\end{itemize}
As\'i mismo, con la herramienta de excel se realiz\'o un an\'alisis de regresi\'on  de los movimientos de animaci\'on y tenis de mesa, tomando en cuenta  la distancia recorrida de la mu\~neca -i.e. Variable independiente- y su respectivo tiempo -i.e. Variable dependiente-, a partir de una muestra de 10 a 15 repeticiones, con la finalidad de comparar los coeficientes de determinaci\'on de las siguientes l\'ineas de tendencias:
\begin{itemize}
\item Exponencial
\item Lineal
\item Logar\'itmica
\item Polinomial de grado 2
\item Polinomial de grado 3
\item Polinomial de grado 4
\item Potencia
\end{itemize}
\subsection{Captura de datos durante una rutina (normalizaci\'on de datos)}\label{dis:recognitionMove}
Durante la rutina, los datos del seguimiento del esqueleto son recuperados a partir del kit de desarrollo de software del Kinect, dichos datos son almacenados en la siguiente estructura:
\begin{itemize}
\item \textbf{I:} Valor num\'erico \'unico que identifica una articulaci\'on.
\item \textbf{X:} Posici\'on horizontal de la articulaci\'on, dibujado en una imagen de dos dimensiones.
\item \textbf{Y:} Posici\'on vertical de la articulaci\'on, dibujado en una imagen de dos dimensiones.
\item \textbf{Fm:} Factor del movimiento proporcionado por la base de datos de gesturas.
\item \textbf{P:} Valor num\'erico que identifica el paso que est\'a realizando un atleta.
\item \textbf{Tiempo:} Tiempo de captura de datos.
\item \textbf{Joint:} Vector que almacena la posici\'on de una articulaci\'on en una figura de dos dimensiones.
\item \textbf{Esqueleto:} Vector que almacena todos los vectores de articulaciones del esqueleto humano.
\item \textbf{Step:} Vector que almacena informaci\'on del seguimiento del esqueleto, factor del movimiento y el tiempo de un paso.
\item \textbf{Repetici\'on:} Vector que almacena la informaci\'on de cada de paso de un movimiento.
\item \textbf{Serie:} Vector que almacena las repeticiones del movimiento durante una serie.
\item \textbf{Series:} Vector que almacena la informaci\'on de cada serie de la rutina.
\end{itemize}
\begin{formula}[H]
	\centering
	\caption{Matriz de datos capturados durante una rutina}
	\label{frm:MatrizDatosRepeticion}
	\begin{equation}
\begin{matrix}
i & enumerador\: de\: la\: articulaci\acute{o}n \\ 
x & distancia\, horizontal \: (p\acute{i}xeles) \\ 
y & distancia\, vertical\: (p\acute{i}xeles) \\ 
joint_{i}& [i,x,y] \\ 
 & \\ 
esqueleto & \begin{bmatrix}
joint_{0} \\ 
... \\ 
joint_{i}\\ 
... \\ 
joint_{24}
\end{bmatrix}  \\ 
 & \\ 
fm & factor \, del \, movimiento \\ 
p  & paso \, del \,movimiento \\ 
t  & tiempo \, total \\ 
step_{p}  & [fm,p,esqueleto, tiempo] \\
 & \\ 
Repeticion & [step_{0}, ...,  step_{col}, ..., step_{last}] \\
serie & [repeticion] \\
series & [serie]
\end{matrix}
	\end{equation}
	\textbf{Fuente:} Propuesto por el autor de tesis
\end{formula}
De igual manera se presenta el algoritmo de captura de datos de la rutina, dicho algoritmo se ejecuta durante el tiempo de trabajo de tabata, en donde comienza con una subrutina para obtener el valor de factor de movimiento en tiempo real, posteriormente ese valor se compara a partir de los intervalos de confianza, si el valor no se encuentra en ning\'un intervalo, se debe obtener un nuevo factor de movimiento (durante un per\'iodo de 0.033 segundos). En caso contrario se obtiene el valor de la etiqueta del paso. Luego se revisa si hay un dato registrado con relaci\'on al paso actual, en caso que haya un registro asociado, se debe eliminar los datos del paso siguiente (esto ocurre, si el usuario se salta un paso del movimiento) y as\'i mismo se elimina el registro del paso actual, despu\'es se continua con el proceso que no haya ning\'un registro del paso, la cual se debe obtener toda la informaci\'on del seguimiento del esqueleto y seguidamente se almacena el registro en el vector de repetici\'on. Finalmente se chequea la cantidad de registros del  vector de repetici\'on, en caso que falte un registro se debe reiniciar nuevamente el algoritmo, por lo contrario, se almacena en el registro de repetici\'on.
\begin{figure}[H]
	\caption{Algoritmo de captura de datos durante la rutina}
	\label{fig:capturaDatos}
	\centering
	\includegraphics[width=430px,height=470px]{graphics/algoritmoDeteccion.png} \\
	\textbf{Fuente:} Elaborado por el autor de tesis
\end{figure}
\subsection{Validaci\'on del modelo de reconocimiento de movimiento}\label{dis:validate}
El presente proyecto se utiliz\'o dos m\'etodos de validaciones cruzadas \cite{perez2015analisis}:
\begin{itemize}
\item \textbf{Hold out:} Las muestras de datos se separan en dos conjunto, uno para construir y entrenar el modelo -i.e. Build- y otro para realizar pruebas que dan validez a los m\'argenes  de errores del modelo.
\item \textbf{K-fold:} Las muestras se dividen en K subconjuntos, y en cada subconjunto se aplica el m\'etodo hold out.
\end{itemize}  
\begin{table}[H]
\begin{center}
\caption{Validaci\'on cruzada, 3-fold de tenis de mesa}
\label{tab:KfoldTenis}
\begin{tabular}{|l|l|l|}
\hline
\multicolumn{3}{|c|}{\textbf{Muestra de v\'ideos}} \\ \hline
\multicolumn{3}{|c|}{1, 2, 3, 4, 5, 6} \\ \hline
\textbf{Modelo} & \textbf{Builds} & \textbf{Test} \\ \hline
1 & 1, 2, 3, 4, 5 & 6 \\ \hline
2 & 1, 2, 4, 5, 6 & 3 \\ \hline
3 & 2, 3, 4, 5, 6 & 1 \\ \hline
\multicolumn{3}{l}{\textbf{Fuente:} Elaborado por el autor de tesis}
\end{tabular}
\end{center}
\end{table}
\begin{table}[H]
\begin{center}
\caption{Validaci\'on cruzada, 3-fold de animaci\'on}
\label{tab:KfoldAnimacion}
\begin{tabular}{|l|l|l|}
\hline
\multicolumn{3}{|c|}{\textbf{Muestra de v\'ideos}} \\ \hline
\multicolumn{3}{|c|}{1, 2, 3, 4, 5, 6, 7} \\ \hline
\textbf{Modelo} & \textbf{Builds} & \textbf{Test} \\ \hline
1 & 1, 2, 3, 4, 5, 6 & 1 \\ \hline
2 & 1, 2, 3, 5, 6, 7 & 4 \\ \hline
3 & 1, 2, 3, 4, 5, 6 & 7 \\ \hline
\multicolumn{3}{l}{\textbf{Fuente:} Elaborado por el autor de tesis}
\end{tabular}
\end{center}
\end{table}
\begin{table}[H]
\begin{center}
\caption{Validaci\'on cruzada, 3-fold de taekwondo}
\label{tab:KfoldTaekwondo}
\begin{tabular}{|l|l|l|}
\hline
\multicolumn{3}{|c|}{\textbf{Muestra de videos}} \\ \hline
\multicolumn{3}{|c|}{1, 2, 3, 4, 5, 6, 7, 8, 9, 10, 11, 12, 13, 14, 15, 16} \\ \hline
\textbf{Modelo} & \textbf{Builds} & \textbf{Test} \\ \hline
1 & 1, 2, 3, 4, 5, 6, 7, 8, 9, 11, 12, 13, 14, 15, 16 & 10 \\ \hline
2 & 1, 2, 3, 4, 5, 6, 7, 8, 9, 10, 12, 13, 14, 15, 16 & 11 \\ \hline
3 & 1, 2, 3, 4, 5, 6, 7, 8, 9, 10, 11, 13, 14, 15, 16 & 12 \\ \hline
\multicolumn{3}{l}{\textbf{Fuente:} Elaborado por el autor de tesis}
\end{tabular}
\end{center}
\end{table}
Posteriormente en la etapa de pruebas, se realiz\'o  por cada modelo una hoja de observaciones de valores reales y pronosticados (ver anexo, tabla \ref{tab:obsErrores}), de modo de encontrar los siguientes errores de pron\'osticos \cite{erroresPronostico}:
\begin{itemize}
\item \textbf{Error medio pronosticado (\acrshort{EMP}):} Promedio de diferencia entre el valor real y pronosticado, la cual puede tener tres interpretaciones distintas:
	\begin{itemize}
	\item \textbf{Valor positivo:} En promedio, los valores pronosticado est\'an por debajo de los valores reales.
	\item \textbf{Valor negativo:} En promedio, los valores pronosticado est\'an por arriba de los valores reales.
	\item \textbf{Valor cero:} Los valores pronosticados pueden estar por debajo o por arriba de los valores reales.
	\end{itemize}
\begin{formula}[H]
	\centering
	\caption{C\'alculo del error medio pronosticado}
	\label{frm:empMath}
	\begin{equation}
EMP=\frac{\sum_{0}^{n}(Real_{x}-Pronostic_{x}\cdot10^{-6})}{n}
	\end{equation}
	\textbf{Fuente:} Formula adaptada al proyecto, a partir de la f\'ormula del error medio pronosticado \cite{videoErrores}
\end{formula}  
\item \textbf{Desviaci\'on media absoluta (\acrshort{DMA}):} Promedio de diferencia absoluta entre el valor real y pronosticado, que muestra la dispersi\'on con respecto al valor real.
 \begin{formula}[H]
	\centering
	\caption{C\'alculo de la desviaci\'on media absoluta}
	\label{frm:MADMath}
	\begin{equation}
DMA=\frac{\sum_{0}^{n}( \left |  Real_{x}-Pronostic_{x}\cdot10^{-6} \right |)}{n}
	\end{equation}
	\textbf{Fuente:} Formula adaptada al proyecto, a partir de la f\'ormula de la desviaci\'on media absoluta \cite{videoErrores}
\end{formula}  
\item \textbf{Ra\'iz del error cuadr\'atico medio (RECM):} Es la desviaci\'on est\'andar de los errores de predicci\'on, la cual indica qu\'e tan disperso se encuentra los errores con respecto al error medio pronosticado.
 \begin{formula}[H]
	\centering
	\caption{C\'alculo de la ra\'iz del error cuadr\'atico medio}
	\label{frm:RECMMath}
	\begin{equation}
RECM=\sqrt{\frac{\sum_{0}^{n}((Real-Pronostic\cdot10^{-6})-EMP)^{2}}{n}}
	\end{equation}
	\textbf{Fuente:} Formula adaptada al proyecto, a partir de la f\'ormula de la ra\'iz del error cuadr\'atico medio \cite{GEORCMETUT}
\end{formula}  
\end{itemize}
Luego de encontrar los errores, se debe seleccionar el modelo que tenga la menor \acrshort{RECM}, y posteriormente encontrar los errores de la muestra total:
 \begin{formula}[H]
	\centering
	\caption{EMP de la muestra total}
	\label{frm:EmpAll}
	\begin{equation}
EMP_{muestra\, total}=AVERAGE([EMP_{modelo}])
	\end{equation}
	\textbf{Fuente:} Propuesto por el autor de tesis, con base a las f\'ormulas de excel (versi\'on en ingl\'es)
\end{formula}  
 \begin{formula}[H]
	\centering
	\caption{MAD de la muestra total}
	\label{frm:MadAll}
	\begin{equation}
MAD_{muestra\, total}=AVERAGE([MAD_{modelo}])
	\end{equation}
	\textbf{Fuente:} Propuesto por el autor de tesis, con base a las f\'ormulas de excel (versi\'on en ingl\'es)
\end{formula}  
	 \begin{formula}[H]
	\centering
	\caption{RECM de la muestra total}
	\label{frm:RecmAll}
	\begin{equation}
RECM_{muestra\, total}=AVERAGE([RECM_{modelo}])
	\end{equation}
	\textbf{Fuente:} Propuesto por el autor de tesis, con base a las f\'ormulas de excel (versi\'on en ingl\'es)
\end{formula} 
Los errores de la muestra total ayudan aceptar o rechazar el modelo; si la \acrshort{RECM} es menor al valor de identificaci\'on (ver f\'ormula \ref{frm:idenStep}), se aprueba. En caso que sea mayor o igual, se rechaza, debido que existe secciones que no podr\'a identificar el paso, tal como se muestra en la figura de aprobaci\'on o rechazo del modelo:
\begin{figure}[H]
	\caption{Aprobaci\'on o rechazo del modelo}
	\label{fig:AproveOrDennie}
	\centering
	\includegraphics[width=430px,height=180px]{graphics/aceptacionModelo.png} \\
	\textbf{Fuente:} Elaborado por el autor de tesis
\end{figure}
Finalmente, por cada modelo aprobado se debe encontrar el factor de reconocimiento y su respectivo rango de confiabilidad:
\begin{formula}[H]
	\centering
	\caption{Factor de reconocimiento e intervalos de confianza}
	\label{frm:rangoConfiabilidad}
	\begin{equation}
\begin{matrix}
recognition=\frac{RECM_{muestra\, total}}{valor\: de\: identificaci\acute{o}n}  \\ 
\\
intervalo\: de\: confianza = ic =[[inferior,superior]] \\
\\
ic(paso)=\left\{\begin{matrix}
inferior(paso)=\left\{\begin{matrix}
0 & si\, es\, el\, primer\, paso\\ \\ 
etiqueta(paso)-\frac{recognition*valor\, de\, identificaci\acute{o}n}{2} & si\: es\: paso\: intermedio \\ 
\\
1-(recognition*valor\, de\, identificaci\acute{o}n)& si\, es\, el\, \acute{u}ltimo\, paso
\end{matrix}\right. \\ 
\\ 
inferior(paso)=\left\{\begin{matrix}
recognition*valor\, de\, identificaci\acute{o}n& si\, es\, el\, primer\, paso \\
\\
etiqueta(paso)+\frac{recognition*valor\, de\, identificaci\acute{o}n}{2} & si\: es\: paso\: intermedio \\ \\ 
1 & si\: es\: el\: \acute{u}ltimo\: paso
\end{matrix}\right.
\end{matrix}\right.
\end{matrix}
	\end{equation}
	\textbf{Fuente:} Propuesto por el autor de tesis
\end{formula} 
\subsection{Dise\~no de los resultados de una rutina tabata} \label{dis:results}
En esta secci\'on se determina los resultados de la rutina de tabata, dichos resultados son encontrados a partir del seguimiento del esqueleto, la cual proporciona informaci\'on de las variables de los detalles del paso y repetici\'on (Ver f\'ormula \ref{frm:MatrizDatosRepeticion}), con el fin objetivo de encontrar la siguiente informaci\'on:
\begin{itemize}
\item \textbf{Volumen de repeticiones:} Sumatoria de las cantidades totales de repeticiones de una serie.  
\begin{code}[H]
	\caption{Pseudoc\'odigo para obtener las repeticiones totales de una rutina}
	\label{code:getRepetitions}
	\begin{lstlisting}
Entradas:
	Series de repeticiones del movimiento
	Contador de Repeticiones

Procesos:
	Recorrer cada serie de repeticiones del movimiento:
		Contar las repeticiones de la serie
		Sumar al contador de repeticiones

Salida
	Contador de repeticiones
	\end{lstlisting}
	\textbf{Fuente:} Elaborado por el autor de tesis
\end{code}

\item \textbf{Duraci\'on:} Tiempo total que se emplea en una rutina, tomando en cuenta todas las series de trabajos y descansos.
\begin{formula}[H]
	\centering
	\caption{C\'alculo de la duraci\'on de tiempo de una rutina}
	\label{eq:DurationTime}
	\begin{equation}
	Duration = \sum_{i=0}^{series}restTime +\sum_{i=0}^{series}workTime = series(restTime+workTime)
	\end{equation}
		\textbf{Fuente:} Elaborado por el autor de tesis
\end{formula}
\item \textbf{Resistencia:} Construye la estructura de la gr\'afica de la resistencia de un atleta durante una rutina a partir del recorrido de todas la series, posteriormente en cada serie se recorre todas las repeticiones y luego en cada repetici\'on se almacena el tiempo acumulado de la rutina y la cantidad de repeticiones que lleva el atleta a ese momento.
\begin{code}[H]
	\caption{Pseudoc\'odigo para crear la gr\'afica de resistencia}
	\label{code:getEndurance}
	\begin{lstlisting}
Entradas:
	Series de repeticiones del movimiento
	Gráfico en donde almacena
		Subgráfico que está compuesto por
			Nombre de la serie
			Listado de puntos en donde cada punto
				Guarda el tiempo (eje x)
				Guarda el numero de repetición (eje y)
	
Proceso:
	Crear gráfico
	Recorrer cada serie de repeticiones del movimiento
		Crear subgráfico
		Obtener el número de serie
		Crear el listado de puntos
		Recorrer cada repetición de la serie
			Obtener el número de repeticiones acumuladas 
			Obtener el tiempo del paso final de la repetición
			Crear un punto
			Guardar punto en el listado de puntos
		Almacenar en el subgráfico el número de serie y listado de puntos
		Guardar subgráfico en el listado que tiene el gráfico

Salida
	Gráfico de resistencia
	\end{lstlisting}
	\textbf{Fuente:} Elaborado por el autor de tesis
\end{code} 

\item \textbf{Potencia:} Resultado que muestra la cantidad m\'axima de repeticiones en el menor tiempo posible, la cual se encuentra a partir del recorrido de las series, luego en cada serie se determina la cantidad de repeticiones totales y el tiempo acumulado de la \'ultima repetici\'on, encontrando: 
	\begin{itemize}
	\item Una mayor cantidad de repeticiones almacenada anteriormente
	\item La misma cantidad de repeticiones almacenada anteriormente, pero se verifica si el tiempo acumulado es menor.
	\end{itemize}
\begin{code}[H]
	\caption{Pseudoc\'odigo para obtener la potencia}
	\label{code:getEndurance}
	\begin{lstlisting}
Entradas
	Series de repeticiones del movimiento
	Potencia en donde almacena
		Cantidad de repeticiones acumuladas de una serie
		Tiempo total de la última repetición de la serie
		
Procesos:
	Crear potencia
	Recorrer cada serie de repeticiones del movimiento
		Obtener la última repetición de la serie
			Obtener el número de repeticiones acumuladas
			Obtener el tiempo del paso final de la repetición
			¿El número de repeticiones acumuladas aumentó?
				Si
					Almacenar los datos en potencia
				Sino si, ¿El número de repeticiones permaneció igual?:
					¿El Tiempo total disminuyó?:
						Si
							Almacenar los datos en potencia
						
Salida
	Potencia 
	\end{lstlisting}
	\textbf{Fuente:} Elaborado por el autor de tesis
\end{code} 

\item \textbf{Velocidad}: Raz\'on de cambio separado por:
	\begin{itemize}
	\item \textbf{Repeticiones por serie:} Variable que se determina a partir del promedio total de repeticiones por serie.
		\item \textbf{Tiempo por repetici\'on:} Variable que se determina a partir del promedio de la diferencia entre el tiempo del paso final y tiempo del paso inicial -i.e. Tiempo de una repetici\'on- de cada repetici\'on realizada por el atleta.
	\end{itemize}
\end{itemize}


\begin{code}[H]
	\caption{Pseudoc\'odigo para obtener las velocidades de las rutinas}
	\label{code:getTimeOfRepetitions}
	\begin{lstlisting}
Entradas
	Series de repeticiones del movimiento
	Velocidad en donde almacena
		Repeticiones por serie
		Tiempo por serie
		
Proceso:
	Crear velocidad
	Crear un listado de tiempos de repeticiones
	Crear un listado de repeticiones de series
	Recorrer cada serie de repeticiones del movimiento
		Obtener el total de repeticiones de la serie
		Agregar al listado de repeticiones de series
		Recorrer cada repetición de la serie
			Obtener el tiempo del paso inicial de la repetición
			Obtener el tiempo del paso final de la repetición
			Encontrar la diferencia de tiempos del paso final e inicial
			Almacenar en el listado de tiempos de repeticiones
	Encontrar el promedio de repeticiones de series
	Transformar a número entero el promedio de repeticiones de series
	Encontrar el promedio de tiempos de repeticiones
	Almacenar ambos promedios en velocidad
	
Salida:
	Velocidad
	\end{lstlisting}
	\textbf{Fuente:} Elaborado por el autor de tesis
\end{code} 
\section{Criterios del proyecto}\label{criter}
En esta secci\'on se presenta todos los criterios que se siguieron para crear el modelo de reconocimientos de movimientos:
\subsection{Criterios de selecci\'on de movimiento}
\begin{itemize}
	\item El investigador y el entrenador de cada deporte  seleccionaron aquellos movimientos que se ejecutan dentro del \'area de visi\'on del sensor Kinect.
	\item El movimiento no debe ser complejo, es decir que cualquier persona pueda realizar dicho movimiento, sin importar su nivel deportivo.
	\item Debe ser un movimiento que se emplea constantemente en el deporte, para aprender nuevos movimientos complejos.
\end{itemize}
\subsection{Criterios de an\'alisis de movimiento}
\begin{itemize}
	\item Cada movimiento debe tener una articulaci\'on de an\'alisis.
	\item El conjunto de articulaciones que interact\'uan en el movimiento, se debe seleccionar con base a la separaci\'on del cuerpo humano, es decir, la parte inferior (Articulaciones abajo de la cadera central) o superior (Articulaciones arriba de la cadera central).
\end{itemize}
\subsection{Criterios de etiquetaci\'on de movimiento}
\begin{itemize}
	\item Cada repetici\'on del movimiento debe pasar por todos los pasos.
	\item El rango de etiquetaci\'on debe estar entre valores de 0 (paso inicial) y 1 (paso final).
	\item La curva de aprendizaje de cada repetici\'on, debe ser similar a la funci\'on sigmoide -i.e. gr\'afica con forma de ese (s)-.
	\item No se etiqueta los datos de ruidos durante una repetici\'on del movimiento -e.g. Interrupciones, errores en el seguimiento del esqueleto-.
\end{itemize}
\subsection{Criterios de error del modelo}
\begin{itemize}
	\item Se debe calcular la dispersi\'on entre los datos de la muestra y su media, a partir de la ra\'iz del error cuadr\'atico medio (RECM).
	\item Se debe calcular la dispersi\'on entre los datos reales y pronosticado a partir de la desviaci\'on media absoluta (DMA). 
	\item El error se encuentra en un rango de 0 a 1.
\end{itemize}
\subsection{Criterios de selecci\'on del modelo}
\begin{itemize}
	\item Se selecciona el modelo que contenga la  menor RECM, en caso que exista RECM iguales, se escoge el modelo que tenga la menor DMA.
\end{itemize}
\subsection{Criterios de aceptaci\'on del modelo}
\begin{itemize}
	\item El modelo seleccionado debe considerar los errores de todas las muestras, a partir del promedio de errores de los submodelos comparados.
	\item Se aprueba el modelo, s\'i el valor promedio de la RECM es menor al valor de  identificaci\'on.
\end{itemize}