\chapter{Discusi\'on}
La interfaz de programaci\'on de aplicaciones del sensor Kinect permite obtener la funcionalidad del seguimiento de esqueleto, lo cual permiti\'o al proyecto generar el conjunto de articulaciones del cuerpo humano, Sin embargo, en el trabajo se observ\'o que dicha funcionalidad ten\'ia fallos, s\'i el usuario no se posicionaba a la distancia de profundidad correcta para renderizar la imagen, lo cual se encontr\'o que la distancia profundidad ideal para generar el seguimiento de esqueleto, se encuentra en un rango de distancia de 2.56 metros a 3.99 metros de profundidad, tomando en cuenta que la altura de sensor al piso es de 0.7 metros.
\medbreak
Adem\'as, el API interact\'ua con otros softwares presentados en el Kit de desarrollo del Kinect (SDK), la cual se utiliz\'o en el proyecto debido a las facilidades de recolecci\'on datos del seguimiento de esqueleto, entre ellas se puede mencionar el software, Kinect Studio, instrumento que monitorea todos los eventos del sensor Kinect. Permitiendo as\'i grabar v\'ideos de atletas realizando rutinas de repeticiones del movimiento, considerando que se seleccion\'o a un grupo de deportistas (Profesionales) para reducir el error de ejecuci\'on de movimiento (Ruido).
\medbreak
Sin embargo, durante las grabaciones de v\'ideos, se observ\'o nuevamente el problema del fallo de seguimiento de esqueleto por nuevas causas, debido a las interrupciones que puede presentar durante un v\'ideo, entre ellos se puede mencionar la detecci\'on de otros seguimientos de esqueletos de otras personas, un ambiente inadecuado -e.g. Exceso de calor, iluminaci\'on incorrecta, lluvia- y el fallo del hardware o software -e.g. Desconexi\'on de la fuente de alimentaci\'on-, por lo tanto para resolver dichos problemas, el investigador repiti\'o nuevamente los v\'ideos en que fallaba el seguimiento esqueleto, buscando reducir el ruido.
\medbreak
En los p\'arrafos anteriores,  se habl\'o de los fallos del seguimiento de esqueleto, sin embargo, en el proyecto se present\'o la generaci\'on incorrecta del seguimiento esqueleto (Funcionaba el seguimiento de esqueleto pero con una mala renderizaci\'on de imagen) durante la ejecuci\'on del movimiento, esto se puede observar al momento de realizar la patada lateral de taekwondo durante la transiciones del paso 2 y 3, ya que el esqueleto no concuerda con la sombra del atleta,  y esto puede ser debido a la posici\'on, altura o direcci\'on de la c\'amara, no obstante, en el proyecto se entren\'o el movimiento con dicho problema, debido que se genera correctamente los pasos 1 y 4.
\medbreak
En cuanto la creaci\'on del modelo de reconocimiento de movimiento, se utiliz\'o el software, Visual Gesture Builder, debido que es una herramienta que permite etiquetar cada fotograma del v\'ideo, as\'i mismo por cada fotograma contiene informaci\'on de movimiento cinem\'atico -e.g. Velocidad, posici\'on, fuerza, torque-, la cual se utiliza para entrenar el modelo con el algoritmo, Random Forest Regression, por lo tanto en el proyecto se etiquet\'o cada  repetici\'on del v\'ideo, con valores de 0 (Paso inicial) y 1 (Paso final),  de modo que en trabajo se etiquet\'o un total de 332 repeticiones de saque de derecha (Tenis de mesa), 1129 repeticiones de Jumping Jack (Animaci\'on) y finalmente 155 repeticiones de patadas laterales (Taekwondo).
\medbreak
Por otro lado, Visual Gesture Builder, tiene una herramienta de an\'alisis que permite compilar el modelo de reconocimiento de movimiento en los v\'ideos, obteniendo el valor pronosticado (Factor de movimiento) y el valor etiquetado de cada fotograma, la cual se utiliz\'o para medir el error del modelo para encontrar la dispersi\'on entre los datos y la media (Ra\'iz del error cuadr\'atico medio) y as\'i mismo la diferencia de los datos reales y pronosticado (Desviaci\'on media absoluta).
\medbreak
Al mismo tiempo, el proyecto ha recolectado poca cantidad de datos para realizar el modelo de reconocimiento de movimiento,  por lo tanto para solucionar dicho problema, el investigador realiz\'o la t\'ecnica de validaci\'on cruzada que permiti\'o crear tres submodelos que son entrenados y testeado con distintos datos de la muestras (Varias combinaciones de datos), y as\'i mismo se verific\'o sus respectivos errores, seleccionando el submodelo que tenga el  menor grado de incertidumbre,  igualmente se calcul\'o el error de toda la muestra, a partir de la media de errores de cada submodelo, la cual ayud\'o a encontrar el valor de reconocimiento para aprobar o rechazar el modelo.
\medbreak
En cuanto a la aprobaci\'on del modelo, el valor de reconocimiento debe ser menor a 1, debido que permitir\'a reconocer correctamente los pasos, ya que si el valor es mayor o igual a 1, los intervalos de confianza se traspasa con los otros intervalos (Provocando un confusi\'on entre los pasos), adem\'as este criterio permiti\'o aprobar los movimiento de saque derecha (Tenis de mesa) y jumping Jack (Animaci\'on). Sin embargo, para la patada lateral (Taekwondo) se rechaz\'o, ya que su valor de reconocimiento es de 1.27, y esto puede ser debido por la poca cantidad de muestra (155 repeticiones de jumping jacks) o el problema de la generaci\'on incorrecta del seguimiento de esqueleto (Paso 2 y 3).
\medbreak
Por lo que se refiere los modelos aceptado, es importante grabar distintas repeticiones con diferentes atletas, ya que puede cambiar los valores de los movimientos cinem\'aticos entre personas (Teniendo variedad de datos para entrenamiento y testeo), y esto se puede observar en las regresiones de distancia contra tiempo de  los movimientos aprobado, ya que cuando el tiempo es de 0.4 segundos, la mu�eca derecha puede recorrer de 0.5 a 0.9 metros durante un saque derecha, mientras que un jumping Jack recorre entre 0.5 a 1.1 metros.
\medbreak
En cuanto a las rutinas deportivas utilizada en el proyecto, se trabaj\'o con los entrenamientos de intervalos controlados,  pongamos por caso que durante la captura de datos, el equipo de animaci\'on utiliz\'o el entrenamiento por tiempo, debido que realizada cantidades de repeticiones de saltos al ritmo de una canci\'on, mientras que para los equipos de taekwondo y tenis de mesa se utiliz\'o el entrenamiento de escaleras infinita (Series repeticiones constantes), ya que se los atletas se le facilitaba trabajar con un objetivo de repeticiones. Por otro lado, para validar los modelos en tiempo real, se trabaj\'o con la rutina Tabata, puesto que una rutina que estandariza el tiempo de trabajo y de descanso en un ejercicio, adem\'as de buscar que el cuerpo metabolize grandes de cantidades de energ\'ia en un per\'iodo corto de tiempo.
\medbreak
Continuando con el proceso de validaci\'on de modelo en tiempo real, el algoritmo reconocer\'a el movimiento, s\'i las transiciones del seguimiento de esqueleto pasan por todos los pasos del movimiento, tal como se muestra en los resultados, un atleta puede realizar en promedio entre 38 a 45 saques derechas durante un tiempo de 40 segundos. Por otro lado, se puede concluir que una persona puede llegar a realizar entre 27 a 30 saltos durante un tiempo de 40 segundos. Sin embargo, cada persona es diferente y puede llegar a realizar una cantidad menor o mayor de repeticiones, pero se debe tomar en cuenta que durante el tiempo trabajo el modelo reconocer\'a repeticiones de movimiento, asegurando que la persona est\'e realizando actividad f\'isica.
\medbreak
Finalizando con la discusi\'on de resultados, se puede afirmar la hip\'otesis y responder a la pregunta general de la investigaci\'on, ya que es posible detectar las repeticiones de un movimiento similares a la muestra de datos de entrenamiento, con la ayuda del modelo de reconocimiento de movimientos, creado por el algoritmo de Random Forest Regression.
