\section{Recomendaciones} \label{ded:rec}
\begin{itemize}
\item Si se trabaja con un movimiento complejo, se recomienda analizarlo por dos o m\'as movimientos simples (divide y conquistar\'a), por ejemplo para el movimiento de patada lateral de taekwondo, se puede analizar por dos movimientos simples: Levantamiento de rodilla a la cadera (paso uno y dos) y levantamiento de pierna a la cadera (paso tres y cuatro).
\item Para disminuir el error del modelo de reconocimiento de los pasos requeridos de un  movimiento v\'alido, se recomienda recolectar m\'as  datos de  repeticiones del movimiento (realizadas por distintos atletas), tal como se muestra en los resultados, el modelo de animaci\'on tiene un error de +/-0.06  con 1229 repeticiones, posteriormente le sigue el modelo de tenis de mesa con un error de +/-0.24 con 332 repeticiones y finalmente el modelo de taekwondo tiene un error de +/-0.32 con 155 repeticiones.
\item Se recomienda realizar un modelo de detecci\'on de los pasos requeridos de un movimiento v\'alido, por cada lugar deportivo, debido que son otros atletas, otros profesionales y otros  criterios para ejecutar un movimiento.
\end{itemize}