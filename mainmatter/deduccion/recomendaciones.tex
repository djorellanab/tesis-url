\section{Recomendaciones} \label{ded:rec}
\begin{itemize}
\item Si se trabaja con un movimiento complejo, se recomienda analizarlo por dos o m\'as movimientos simples (divide y conquistar\'a), por ejemplo para el movimiento complejo burpee, se puede analizar por dos movimientos simples: Lagartija -i.e. push up- y salto.
\item Para aumentar la precisi\'on y disminuir el error del reconocimiento de movimiento, es necesario m\'as datos de  repeticiones del movimiento (realizadas por distintos atletas), lo cual se aconseja ir a distintos grupos deportivos que realizan dicho movimiento -e.g. Universidades, federaciones, colegios, gimnasios-.
\item Se sugiere capturar los datos en un  lugar que tenga un clima agradable -e.g. Sin lluvia o exceso de sol-, con un espacio suficiente para realizar el movimiento sin ninguna interrupci\'on -e.g. Lugares planos-, adem\'as de tener una  ventilaci\'on e iluminaci\'on correcta -e.g. Lugares que no se acumule  el calor-.
\item Es recomendable para el atleta realizar una rutina de calentamiento con la vestimenta adecuada, para preparar el organismo y evitar alguna lesi\'on durante el entrenamiento.
\item Para futuros proyectos se recomienda implementar un modelo que clasifique una repetici\'on como bueno o malo.
\item Es aconsejable que cada grupo deportivo almacene los resultados de sus atletas, para ver si el sedentarismo de cada deportista ha mejorado o empeorado en un lapso de tiempo.
\end{itemize}