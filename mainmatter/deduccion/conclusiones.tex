\section{CONCLUSIONES} \label{ded:con}
\begin{itemize}
\item De acuerdo a los tres deportes estudiados, se puede analizar movimientos que son f\'aciles de hacer dentro el campo de visi\'on del sensor Kinect.
\item Conforme a los tres movimientos analizados, se puede deducir que un movimiento esta compuesto por dos o m\'as pasos, y en cada paso se debe ejecutar  la postura correcta.
\item En relaci\'on a los entrenamientos de los deportes seleccionados, se puede concluir que el calentamiento es una actividad importante para la preparaci\'on de los movimientos que se ejecutan en una rutina.
\item Acorde a las rutinas de captura de datos utilizados en el proyecto,  se puede inferir que la rutina por tiempo, es utilizado en los atletas que puede ejecutar un mayor volumen de repetici\'on en un largo per\'iodo de tiempo (Sin importar la cantidad de repeticiones), mientras que la rutina por repeticiones (Escaleras), se establece el n\'umero de repeticiones que debe ejecutar el deportista.
\item  Con respecto al seguimiento de esqueleto de cada movimiento analizado, se puede deducir que la distancia correcta para que el sensor Kinect pueda detectar al atleta, es de  2.56 metros a 3.99 metros de profundidad. Sin embargo, el sensor debe estar a una altura de 0.70 metros (Respecto al suelo).
\item  En base a las grabaciones (Tomadas por la herramienta, Kinect Studio) durante la etapa de recolecci\'on de datos, se puede concluir que existe tres tipos de sucesos que puede generar datos inservible; la primera consta de las interrupciones de personas u objetos, adem\'as de los eventos inesperado, como la lluvia. As\'i mismo el segundo suceso se relaciona con las fallas del hardware o software que impide visualizar el seguimiento del esqueleto y finalmente el tercer evento consta en la grabaci\'on dentro de un entorno no ideal, pongamos por caso los espacios abiertos y no planos.
\item En relaci\'on a la implementaci\'on de la base de datos de reconocimiento de gesturas y posturas, se puede deducir que el software, Visual Gesture Builder (Herramienta de inteligencia artificial del SDK del Kinect), crea modelos de reconocimiento a partir de la etiquetaci\'on decimal de fotogramas, es decir por cada paso del movimiento analizado esta etiquetado con un n\'umero entre 0 a 1.
\item De acuerdo a los modelos de pron\'ostico de gesturas y  posturas, se puede concluir que el valor resultante de cada modelo se le llama, factor de movimiento, la cual puede ser comparado por el valor etiquetado de las grabaciones de pruebas, con el fin de objetivo de encontrar los errores del pron\'ostico (Diferencia entre factor movimiento y valor etiquetado).
\item Conforme a los errores de pron\'ostico de cada modelo, se determin\'o el error promedio (desviaci\'on media absoluta), as\'i mismo se construy\'o los rangos de confianza de cada paso, con la ayuda de la desviaci\'on t\'ipica del modelo (Ra\'iz del error cuadr\'atico medio).
\item Por lo que se refiere a la interfaz gr\'afica del seguimiento de esqueleto, se ha utilizado la tecnolog\'ia Windows Presentation Foundation, dado  las herramientas de dibujos (Windows media), manejos de temporizadores (Windows Threading) y manejos de los datos del sensor (SDK de Kinect).
\item En cuanto al algoritmo de detecci\'on de repetici\'on de un movimiento, se detecta \'unicamente, si el seguimiento de esqueleto del atleta pasa por todos los pasos del movimiento de manera ordenada y ascendente.
\item Por lo que se refiere a los resultados de los entrenamiento de cada movimiento estudiado, se ha deducido que la rutina Tabata permite contabilizar las repeticiones del movimiento durante los tiempos de trabajos, adem\'as de ser un entrenamiento de alta intensidad con alta metabolizaci\'on de energ\'ia (Obtenci\'on y gasto de mol\'eculas de ATP) en todo el cuerpo, durante los tiempos de trabajo y descanso.
\item De acuerdo a la comprobaci\'on del modelo de reconocimiento de cada movimiento, se ha comprobado y validado cada modelo por medio de la validaci\'on  cruzada, 3-Fold, la cual permite crear distintos modelos de reconocimiento de movimiento a partir de  combinaciones de datos de pruebas y entrenamientos.
\end{itemize}