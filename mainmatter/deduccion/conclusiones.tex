\section{Conclusiones} \label{ded:con}
\begin{itemize}
\item De acuerdo a los tres deportes estudiados, se pueden concluir que el profesional (del lugar deportivo) estandariza los pasos requeridos para realizar un movimiento v\'alido.
\item Conforme a las grabaciones de los v\'ideos, se debe respetar las distancias de profundidad entre el atleta y el usuario para detectar el seguimiento del esqueleto, adem\'as de estar en un ambiente ideal (ventilaci\'on, espacios planos y  abiertos, clima agradable para realizar el entrenamiento y conexiones electr\'icas).
\item La desviaci\'on del factor del movimiento con respecto a la etiqueta (RECM), es un error que  permite construir los intervalos de confianza para reconocer los pasos requeridos de un movimiento.
\item La certeza de los modelos de detecci\'on  de los pasos requeridos de un movimiento, est\'a dado por su valor recognition, la cual se busca un valor  cercano al 100\%, ya que los intervalos de confianza reconocer\'a fotogramas parecido a cada paso.
\item Se concluye que el algoritmo clasifica como un movimiento v\'alido, si el atleta pasa por todos los pasos de manera ordenada, en caso que se salte un paso, se clasifica como un movimiento inv\'alido.
\item La certeza del algoritmo clasificador de un movimiento v\'alido est\'a dado por sus porcentajes de detecci\'on v\'alidos e inv\'alidos, adem\'as de detallar los porcentajes de falla de cada paso inv\'alido.
\item Se concluye que el algoritmo clasificador busca aumentar el porcentaje de detecci\'on de movimientos v\'alidos con respecto al porcentaje de referencia v\'alidos y por ende busca disminuir el porcentaje de detecci\'on inv\'alidas con respecto al porcentaje de referencia inv\'alidas.
\end{itemize}