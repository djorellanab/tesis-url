%index
\chapter{Introducci\'on}
Durante los a\~nos 1990 a 2016, la Organizaci\'on Mundial de la Salud ha recopilado datos sobre los distintos riesgos met\'abolicos que han afectado a distintos pa\'ises, lo cual ha encontrado que alrededor de 1.6 millones de personas mueren anualmente por no realizar la suficiente actividad f\'isica. Por lo tanto, la inactividad f\'isica es un problema que afecta hoy en d\'ia y principalmente en Guatemala, ya que en el a\~no 2006, se realiz\'o una encuesta de enfermedades no transmisibles en la comunidad de Villa Nueva, Guatemala, la cual concluyeron que alrededor de 27.68\% de los guatemaltecos, presenta sedentarismo por no realizar ninguna actividad f\'isica.
\medbreak
Debido a este problema, la Organizaci\'on de las Naciones Unidas crea el tercer objetivo de desarrollo sostenible, salud y bienestar de las personas, la cual busca reducir las muertes prematuras por enfermedades no transmisible, a partir de las nuevas tecnolog\'ias y educaci\'on. Por esto, la Organizaci\'on Mundial de la Salud impulsa el movimiento: "Por tu salud, mu\'evete", y a ra\'iz de esto, en el a\~no 2018, el Ministerio de Salud P\'ublica y Asistencia Social de Guatemala impulsa el lema: "Salud para todos y todas, en todas partes".
\medbreak
Partiendo de este lema, se ha creado este proyecto de ingenier\'ia, la cual se analizar\'a la interfaz de programaci\'on de aplicaciones del dispositivo Kinect, en base al funcionamiento del seguimiento de esqueleto de un movimiento, adem\'as se utilizar\'a el kit de desarrollo (SDK) para grabar v\'ideos del seguimiento de esqueleto de un atleta realizando una cantidad de repeticiones del movimiento, por medio del software Kinect Studio, y posteriormente etiquetar cada fotograma del v\'ideo para crear un modelo de reconocimiento de movimiento, con la ayuda del software, Visual Gesture Builder.
\medbreak
Al mismo tiempo, el modelo de reconocimiento de un movimiento seleccionar\'a diferentes atletas de la Universidad Rafael Land\'ivar, con el objetivo de obtener datos de entrenamientos y testeos. Por otro lado, se validar\'a el modelo por medio de la t\'ecnica de validaci\'on cruzada, en donde se crear\'a tres submodelos distintos que ser\'an entrenados por el algoritmo, bosques de regresiones aleatorias. As\'i mismo, por cada submodelo se calcular\'a los errores, de modo que se seleccionar\'a el submodelo que contenga el menor error de incertidumbre.
\medbreak
Por otro parte, el submodelo seleccionado tomar\'a en cuenta los errores de la muestra total a partir de los promedios de errores de cada submodelo comparado, adem\'as de verificar que reconozca todos los pasos que debe seguir el movimiento.
\medbreak
En resumen, este proyecto aportar\'a modelos de reconocimiento de tres movimientos distintos practicados en diferentes deportes, con la finalidad de verificar que una persona est\'e realizando actividad f\'isica a partir de la detecci\'on de un movimiento.
\section{LO ESCRITO SOBRE EL TEMA}
\lipsum[1-2]
\section{MARCO TE�RICO}
\subsection{C\'AMARA CON SENSOR DE PROFUNDIDAD}
Los autores, \citeA{carfagni2017performance}, realizaron un estudio del rendimiento de la  c�mara, Intel SR300, en dicho estudio  se�alan las principales funciones de las \acrfull{RGBD}, entre ellas se puede mencionar la adquisici�n y procesamientos de datos en \acrfull{TRESD}, cabe mencionar que estas c�maras se han utilizado en el sector industrial y acad�mico (e.g. Reconocimiento de posiciones, gestos y objetos).
\medbreak
Por otro lado los autores, \citeA{henry2012rgb}, investigaron las variables necesarias para realizar un mapeo del ambiente (i.e. RGB-D mapping), cabe mencionar que las variables son analizado por medio de \gls{pixeles} de informaci�n de im�genes a color y de profundidad, tal como se muestra en la siguiente figura:
\begin{figure}[H]
	\caption{Captura de datos de una c\'mara con sensor de profundidad}
	\label{fig:RGBD}
	\centering
	\includegraphics[width=220px,height=50px]{graphics/RGB-D.PNG} \\
	\textbf{Fuente:} Tomado por el autor de tesis
\end{figure}
La figura \ref{fig:RGBD}, fue capturado por el dispositivo, Kinect de XBox One, a una velocidad de 28 \acrfull{FPS}, cabe mencionar que de lado izquierdo se tiene una vista a color, mientras que del lado derecho se tiene una vista de escala de grises (i.e. Datos de profundidad), por lo tanto en ambas im�genes se puede capturar distintas variables de an�lisis (e.g. Distancia, seguimiento, color, entre otras variables m�s...).
\subsubsection{Dispositivos en el mercado}
A continuaci�n se presentar� una lista de c�maras con sensor de profundidad, que se encuentra en el mercado hoy en d�a:
\begin{itemize}
	\item \textbf{ASUS XtionPro Live:} la corporaci�n, 	 ASUSTeK Computer \cite{xtionAsus}, desarroll� una c�mara con sensor de profundidad e infrarrojo, que permite detectar profundidad adaptativa, imagen a color y flujo de audio.
	\item \textbf{Structure Sensor:} La empresa, Occipital inc \cite{structureOccipital}, implement� un sensor de estructura, que permite escanear las personas, los espacios y los objetos en 3D. 
	\item \textbf{Intel RealSense cameras:} La empresa \citeA{intelRealSense}, desarroll� un producto de c�maras (i.e. Intel RealSense) que permite detectar una alta velocidad de cuadros (i.e. Frames por segundo), RGB de calidad y una resoluci�n de profundidad, cabe mencionar que estas c�maras se han utilizado en soluciones innovadoras (e.g. R�botica, drones, realidad virtual, entre otras aplicaciones m�s).
	\item \textbf{Microsoft Kinect:} El autor	\citeA{zhang2012microsoft}, realiz� un an�lisis de los componentes del Kinect, entre ellos se encuentra el sensor de profundidad, una c�mara a color y una matriz de micr�fonos que permite capturar movimientos, reconocer caracter�sticas faciales, construir un modelo del cuerpo en 3D y reconocer sonidos.
\end{itemize}
Tal como se observa en el listado, todas las c�maras con RGB-D tiene caracter�sticas similares que permite detectar elementos en el ambiente, sin embargo a la hora de escoger una c�mara se debe tomar en cuenta las siguientes especificaciones:
\begin{figure}[H]
	\caption{Especificaciones de una c\'amara con RGB-D}
	\label{fig:RGBESP}
	\centering
	\includegraphics[width=300px,height=170px]{graphics/RGBFeatures.png} \\
	\textbf{Fuente:} Tomado por el autor de tesis
\end{figure}
\medbreak
En la figura \ref{fig:RGBESP}, se muestra una vista general de las caracter�sticas de una c�mara con RGB-D, entre ellas se puede observar el alcance m�ximo de profundidad (i.e. Alcance del sensor 3D). As� mismo se encuentra el \acrfull{FOV}, cuya finalidad es determinar el �ngulo m�ximo de visi�n, respecto a los ejes: horizontal(H), vertical (V) y profundidad (D). Por otra parte, se encuentra la conexi�n de la c�mara al dispositivo, cuya funci�n es capturar los datos de entradas y salidas, como por ejemplo: la Resoluci�n de colores e infrarrojo (Tal como se observa en la figura \ref{fig:RGBD}).
\medbreak
A continuaci�n se presentar� una comparaci�n de las especificaciones entre las c�maras de RGB-D, mencionadas anteriormente:
\begin{table}[H]
\begin{center}
\caption{Comparaci�n de especificaciones entre c�maras de RGB-D }
\label{tab:RGBD}
\begin{tabular}{|l|l|l|l|l|l|} 
\hline
\textbf{Caracter�sticas}                                                          & \begin{tabular}[c]{@{}l@{}}\textbf{ASUS}\\\textbf{XtionPro}\\\textbf{Live}\end{tabular}   & \begin{tabular}[c]{@{}l@{}}\textbf{Structure}\\\textbf{Sensor}\end{tabular}  & \begin{tabular}[c]{@{}l@{}}\textbf{Intel}\\\textbf{RealSense}\\\textbf{SR300}\end{tabular}  & \begin{tabular}[c]{@{}l@{}}\textbf{Microsoft}\\\textbf{Kinect}\\\textbf{Live}\end{tabular} & \begin{tabular}[c]{@{}l@{}}\textbf{Microsoft}\\\textbf{Kinect}\\\textbf{v2}\end{tabular}	  \\ 
\hline
\begin{tabular}[c]{@{}l@{}}\textbf{Alcance del}\\\textbf{ sensor 3D}\end{tabular} & 0.8 a 3.5 m                                                                               & 0.4 a 3.5m                                                                   & 0.2 a 1.5m                                                                                  & 1.8 a 3.5m                                                                                 & 1.3 a 3.5m                                                                                  \\ 
\hline
\textbf{3D Resoluci�n}                                                            &\begin{tabular}[c]{@{}l@{}}640x480\\30fps\end{tabular}                   & \begin{tabular}[c]{@{}l@{}}640x480\\30fps\end{tabular}     & \begin{tabular}[c]{@{}l@{}}640x480\\60fps\end{tabular}                    &\begin{tabular}[c]{@{}l@{}}320x240\\30fps\end{tabular}					& \begin{tabular}[c]{@{}l@{}}512x424\\30fps\end{tabular}                    \\ 
\hline
\begin{tabular}[c]{@{}l@{}}\textbf{RGB}\\\textbf{ Resoluci�n}\end{tabular}        &\begin{tabular}[c]{@{}l@{}}1280x1024\\30fps\end{tabular}                   & \begin{tabular}[c]{@{}l@{}}640x480\\30fps\end{tabular}     & \begin{tabular}[c]{@{}l@{}}1920x1080\\30fps\end{tabular}                    &\begin{tabular}[c]{@{}l@{}}640x480\\30fps\end{tabular}					& \begin{tabular}[c]{@{}l@{}}1920x1080\\30fps\end{tabular}                    \\ 
\hline
\textbf{FOV}                                                                      & 58�H, 45�V                                                                                & 58�H, 45�V                                                                   & 73�H, 59�V                                                                                  & 57�H, 43�V                                                                                 & 70�H, 60�V                                                                                  \\ 
\hline
\textbf{Conexi�n}                                                                 & USB 2.0                                                                                   & USB 2.0                                                                      & USB 3.0                                                                                     & USB 2.0                                                                                    & USB 3.0                                                                                     \\
\hline
\end{tabular}
\end{center}
\textbf{Fuente:} Desarrollo de una aplicaci�n interactiva con Intel RealSense \cite{molero2018desarrollo} y Evaluation of the spatial resolution accuracy of the face tracking system for kinect for windows v1 and v2 \cite{amon2014evaluation}
\end{table}
\medbreak
En la tabla \ref{tab:RGBD}, se puede determinar que la c�maras: Intel RealSense SR300 y Microsoft Kinect V2, destacan de las dem�s c�maras, debido  que tiene una mayor resoluci�n del sensor RGB y un campo de visi�n m�s amplio, por lo cual para el presente proyecto se seleccionar� la c�mara Microsoft Kinect V2. 
\subsubsection{Microsoft Kinect V2}
En la gu�a de programaci�n de Kinect para Windows SDK \cite{jana2012kinect}, da a conocer caracter�stica del sensor Kinect, entre ellas se puede mencionar que el sensor Kinect se desarroll� para la consola de videojuegos, Xbox 360, adem�s proporciona una \acrfull{NUI}, que permite interactuar con el dispositivo a partir de movimientos, gestos y sonidos (i. e. Tecnolog�a con control de manos libres).
\medbreak
Por otra parte el autor, \citeA{jana2012kinect}, habla sobre el \acrfull{SDK}, cabe mencionar que dicho kit esta desarrollado para distintos lenguajes de programaci�n (e.g. C++, c\#, Python), sin embargo, para entender el funcionamiento del SDK se debe conocer la arquitectura del Kinect.
\\
\paragraph{Componentes}\mbox{} \\
\begin{figure}[H]
	\caption{Componentes del Kinect V2}
	\label{fig:COMPKINECT}
	\centering
	\includegraphics[width=400px,height=220px]{graphics/kinect-parts.PNG} \\
	\textbf{Fuente:} Tomado por el autor de tesis
\end{figure}
\medbreak
En la figura \ref{fig:COMPKINECT}, se muestra los componentes del Kinect que enlista el autor \citeA{jana2012kinect}, entre ellas se puede observar la c�mara a color cuya funci�n es capturar y transmitir los datos de v�deo en color, as� mismo se encuentra el sensor de profundidad conformado por el emisor de infrarrojo, que se encarga de escanear el ambiente constantemente  y convertirlo en informaci�n a partir del sensor de profundidad infrarroja (i.e. Identificaci�n de objetos), cabe mencionar que la c�mara puede rotar la imagen a partir de un peque�o motor que conecta la base y el cuerpo. Por otro lado, la matriz de micr�fonos permite capturar e identificar la direcci�n del sonido en el ambiente. Finalmente, se encuentra el  \acrfull{LED}, cuya tarea es identificar si los controladores del Kinect (e.g. IR, Seguimiento de objetos, RGB y sonido) est�n funcionando correctamente a partir de una luz blanca.
\paragraph{Conexi�n a la computadora}\mbox{} \\
\begin{figure}[H]
	\caption{Adaptador del Kinect V2}
	\label{fig:ADAPTERKINECT}
	\centering
	\includegraphics[width=400px,height=170px]{graphics/adapter-kinect.jpg} \\
	\textbf{Fuente:} \citeA{Kinectmanual}
\end{figure}
\medbreak
Tal como se observa en la figura \ref{fig:ADAPTERKINECT}, el cable de conexi�n esta conformado por 5 partes: (1) el cable de datos del Kinect que se encarga de recibir los datos de entradas y salidas del sensor, as� mismo esta (2) al adaptador del Kinect, que permite administrar los datos de entradas y salidas de la computadora y el sensor, a partir del (3) cable de USB 3. Cabe mencionar que dicho adaptador funciona con  un voltaje de 12 voltios y una corriente de 3 amperios que son administrado de una (4-5) fuente de alimentaci�n  que regula una entrada de 100 a 240 voltios y una corriente de 1.6 amperios.
\paragraph{Kit de desarrollo de Software (SDK)}\mbox{} \\
Para el presente proyecto se utilizar� el SDK de la empresa  \citeA{SDKKinect}, este SDK permite crear aplicaciones de reconocimientos de gestos y de voz con el sensor Kinect.
Cabe mencionar que para realizar estas aplicaciones, es necesario entender la interacci�n del software y del hardware:
\begin{figure}[H]
	\caption{Interacci�n del software y hardware}
	\label{fig:interaccionKinect}
	\centering
	\includegraphics[width=400px,height=180px]{graphics/interacionKinect.png} \\
	\textbf{Fuente:} Realizado por el autor de tesis
\end{figure}
En la figura \ref{fig:interaccionKinect}, se puede observar que el sensor kinect maneja 3 transmisiones de datos de salida:
\begin{itemize}
	\item \textbf{Datos de la imagen a color:} Seg�n el estudio de los c�digos de fuentes del SDK del Kinect \cite{hernandez2013analisis}, mencionan que los datos de imagen a color trabajan con un nivel de calidad que determina la velocidad en que los datos son transferidos (i.e. Frames por segundos), por otro lado permite conocer el formato en que se esta enviando dicha informaci�n: RGB (i.e. Mapa de bits a color de 32 bits) o YUV (i.e. Mapa de bits a color de 16 bits con correcci�n de transparencia de imagen).
	\item \textbf{Datos de c�mara de profundidad:} Los autores, \citeA{hernandez2013analisis}, mencionan que dicha transmisi�n esta conformada por el sensor de profundidad, cuya tarea es almacenar una escala de grises de todo el campo visible en un conjuto de p�xeles que representa una distancia de cercan�a con la c�mara (i.e. Base, altura y profundidad), cabe mencionar que la informaci�n es almacenado en 2 Bytes, la cual un Byte corresponde al emisor de IR y el otro Byte corresponde al sensor de profundidad IR.
	\item \textbf{Datos del sonido:} Seg�n el libro de detecci�n de movimiento y profundidad para NUI \cite{rahman2017beginning}, habla sobre la transmisi�n de sonido, dicha transmici�n captura el sonido en un rango m�ximo de 180 grados, as� mismo la informaci�n es almacenada en un vector de Byte (e.g. WAVEFORMAT, estructura b�sica de 16 Bytes).
\end{itemize}
Cabe mencionar que todas las transmisiones  interact�an a trav�s del recurso, NUI, dicho recurso esta compuesto por los siguientes elementos:
\begin{figure}[H]
	\caption{Arquitectura del Software, Kinect-NUI}
	\label{fig:architecturesSoftwareKinect}
	\centering
	\includegraphics[width=390px,height=300px]{graphics/kinect-software-architecture.PNG} \\
	\textbf{Fuente:} \cite[p.~14]{giori2013kinect}
\end{figure}
En el Libro, Kinect en  movimiento \cite{giori2013kinect}, dar a conocer los elementos que trabajan en el NUI (i.e. figura \ref{fig:interaccionKinect}):
\begin{enumerate}[1.]
    \item \textbf{Kinect Sensor:} Elemento que administra la conexi�n y los componentes del hardware.
    \item \textbf{Kinect Drivers:} Elemento que administra los drivers necesarios para el funcionamiento del Kinect (e.g. Audio, Arreglos de audio, c�maras, dispositivos y seguridad), cabe mencionar que dichos driver son accesible en el directorio:  \%Windows\%/System32/DriverStore/FileRepository, dentro de la carpeta "kinectsensor.inf".
    \item \textbf{NUI API:} Elemento que administra los componentes de SDK (i.e. Rastreo de esqueleto, audio, imagen de profundidad y de color), cabe mencionar que dichos componentes son accesible en el directorio: \%Archivos de programa\%/Microsoft SDKs/Kinect
    \item \textbf{DirectX Media Object (DMO):} Elementos que administra el funcionamiento de la matriz de audio (e.g. Identificar y analizar el origen de la fuente del audio).
    \item \textbf{Windows Standar API:} Elementos que administra los componentes complementarios del funcionamiento del sensor (e.g. Microsoft.speech, System.media, etc...).
\end{enumerate}
\subparagraph{Kinect v2 configuration verifier
}\mbox{} \\
Dicha aplicaci�n verifica y analiza el dispositivo que se encuentra conectado al sensor Kinect, tomando en cuenta las compatibilidades del hardware y la comunicaci�n del dispositivo al sensor:
\begin{figure}[H]
	\caption{Kinect v2 configuration verifier}
	\label{fig:KinectConfigurationVerifier}
	\centering
	\includegraphics[width=360px,height=300px]{graphics/kinect-configuration-verifier.PNG} \\
	\textbf{Fuente:} Tomado por el autor de tesis
\end{figure}
\medbreak
En la figura \ref{fig:KinectConfigurationVerifier}, se observa las siguientes validaciones:
\begin{itemize}
	\item \textbf{Update Configuration Definitions:} Verifica que tenga la �ltima versi�n del SDK (e.g. V.2).
		\item \textbf{Operating System:} Verifica si el sistema operativo es compatible (e.g. Windows 8 o superior).
		\item \textbf{Processor Cores:} Detecta si el procesador tiene los suficientes n�meros de core (e.g. Intel 5).
		\item \textbf{Physical Memory (RAM):} Chequea si el dispositivo tiene la suficiente memoria (e.g. M�nimo 4GB de RAM).
		\item \textbf{Graphics Processor:} Verifica si el procesador gr�fico es compatible con el SDK. (e.g. Directx 11).
		\item \textbf{USB Controller:} Verifica si el dispositivo reconoce el puerto de entrada (i.e. USB 3).	
		\item \textbf{Kinect Connected:} Detecta si el Sensor Kinect se encuentra conectado con el dispositivo.
		\item \textbf{Verify Kinect Software Installed:} Verifica las unidades (Drives) del sistema y el sensor.
		\item \textbf{Verify Kinect Depth and Color Streams:} Verifica los sensores de profundidad y de color (i.e. RGB e IR).
\end{itemize}
\subparagraph{Kinect Studio}\mbox{} \\
En el libro de detecci�n de movimiento y profundidad para NUI \cite{rahman2017beginning}, trabajan con algunas aplicaciones ya implementadas en el SDK, entre ellas se puede mencionar, Kinect Studio, aplicaci�n que permite capturar y reproducir datos del sensor en formato de v�deo, cabe mencionar que los datos son administrados por los siguientes monitores:
\begin{itemize}
	\item \textbf{Monitor NUI Body Frame:} Transmisi�n que contiene un espacio de almacenamiento para trabajar los datos de articulaciones del cuerpo.
	\item \textbf{Monitor NUI Body Index:} Transmisi�n que se encarga de clasificar y determinar los p�xeles de cada objeto.
	\item \textbf{Monitor NUI Depth:} Transmisi�n que se encarga de trabajar los datos de profundidad (i.e. Eje Z).
	\item \textbf{Monitor NUI IR:} Transmisi�n que trabaja con una imagen infrarroja a partir de la t�cnica de \gls{TOF}.
	\item \textbf{Monitor NUI Title Audio:} Transmisi�n que suministra el audio capturado en todas las direcciones.
	\item \textbf{Monitor NUI uncompressed color:} Transmisi�n encargada de proporcionar los datos de la imagen a color.	
\end{itemize}
Estos datos son almacenados en un archivo con formato, \acrfull{XEF}, dicho formato genera un v�deo de gran tama�o debido a la cantidad de datos que son capturados por los monitores, por lo cual se recomienda realizar varias repeticiones del movimiento en el menor tiempo posible y de igual manera capturar los siguientes datos para el an�lisis del movimiento:
\medbreak
\begin{figure}[H]
	\caption{Diagrama de Venn para la identificaci�n del movimiento de un objeto}
	\label{fig:VennStreaming}
	\centering
	\includegraphics[width=220px,height=150px]{graphics/venn-streaming.png} \\
	\textbf{Fuente:} Realizado por el autor de tesis
\end{figure}
\medbreak
En la figura \ref{fig:VennStreaming}, se puede observar que los monitores: Body index, Depth e IR, trabajan conjuntamente para reconocer los objetos, as� mismo es importante obtener datos correctos por medio de la calibraci�n de datos y tener disponible los datos en todo momento (i.e. Telemetr�a).
\medbreak
Finalmente en la figura \ref{fig:PlayKinectStudio}, puedes ver el resultado de un v�deo tomado por la aplicaci�n Kinect Studio, cabe mencionar que Kinect Studio te proporciona varias herramienta  para analizar el v�deo, tales como: el punto de Inicio y Fin, cuya funci�n es marcar una interacci�n (i.e. Segmento del v�deo), as� mismo puede indicar el n�mero de repeticiones de la interacci�n (i.e. cantidad de veces que desea ver el segmento del v�deo), adem�s de colocar puntos de pausa y etiquetas de metadatos, que te permitir� almacenar informaci�n adicional en un punto espec�fico del v�deo.
 \begin{figure}[H]
	\caption{Visualizaci�n del v�deo, Kinect Studio}
	\label{fig:PlayKinectStudio}
	\centering
	\includegraphics[width=400px,height=280px]{graphics/play-kinectstudio.png} \\
	\textbf{Fuente:} Tomado por el autor de tesis
\end{figure}
\subparagraph{Visual Gesture Builder}\mbox{} \\
El libro de detecci�n de movimiento y profundidad para NUI \cite{rahman2017beginning}, trabaja con una aplicaci�n de aprendizaje autom�tico llamada: \acrfull{VGB}, que permite crear una base de datos que reconocen los gestos en tiempo de ejecuci�n (Gesture DataBase, GDB), cabe mencionar que la herramienta utiliza dos algoritmos para la detecci�n de objetos, \acrfull{RFR} para modelos continuos, y \acrfull{AdaBoost} para modelos discretos. Se debe tomar en cuenta que los algoritmos analiza las siguientes variables en funci�n del tiempo:
\begin{itemize}
	\item Diferencia de posiciones.
	\item �ngulos de articulaciones y de movimientos.
	\item Velocidad de desplazamiento y angular.
	\item Aceleraci�n de desplazamiento y angular.	
	\item Fuerza muscular
	\item Torque muscular
\end{itemize}
Por lo que se refiere al modelo AdaBoost el autor, \citeA{AdaBoosting2018}, p�blica un art�culo cient�fico sobre la t�cnica de Boosting, en la cual consta en mejorar las predicciones del modelo, a partir de un n�mero de entrenamientos secuenciales, tal como se observa la siguiente figura:
\begin{figure}[H]
	\caption{T�cnica Boosting}
	\label{fig:AdaBoost}
	\centering
	\includegraphics[width=400px,height=200px]{graphics/AdaBoosting.png} \\
	\textbf{Fuente:} Realizado por el autor de tesis
\end{figure}
\medbreak
En la figura \ref{fig:AdaBoost}, se observa que en el entrenamiento \# 1,  los datos de entradas se encuentra muy dispersos, por lo tanto, cada dato de entrada se  debe entrenar a partir de un algoritmo de aprendizaje, que permitir� transformar una nueva funci�n, tal como se observa el entrenamiento \# 2, los datos de entradas est�n m�s cercano. Cabe mencionar que se debe establecer los n�meros de entrenamientos y el algoritmo de aprendizaje que utilizar� por cada entrenamiento, siguiendo con el ejemplo de la figura \ref{fig:AdaBoost}, tiene un total de 2 entrenamientos, y cada entrenamiento consta de operar el dato de entrada por su respectiva regresi�n lineal y una constante.
\medbreak
Cabe mencionar que el modelo discreto, es recomendable utilizarlo para identificar movimientos est�ticos (e.g. Identificar si la persona se encuentra sentada, arrodillada, entre otros movimientos m�s...), por lo que el software, VGB, permite analizar el v�deo (i.e. xef) y posteriormente etiquetar los momentos que se encuentra realizando dicho movimientos est�ticos:
\begin{figure}[H]
	\caption{Etiquetas de movimientos est�tico}
	\label{fig:modeloDiscreto}
	\centering
	\includegraphics[width=400px,height=350px]{graphics/modelo-discreto.png} \\
	\textbf{Fuente:} Realizado por el autor de tesis
\end{figure}
\medbreak
En la figura \ref{fig:modeloDiscreto}, se puede observar que en el panel de control puede etiquetar los valores: positivos (Barra azul arriba) y negativos (Barra azul abajo), as� mismo se observa el movimiento est�tico, Manos abajo, que consta en identificar que ambas manos y brazos est�n tocando la parte dorsal del cuerpo (i.e. Figura \ref{fig:modeloDiscreto}.B), en caso que este realizando otro movimiento, el valor de la etiqueta ser� falso (i.e.  Figura \ref{fig:modeloDiscreto}.C).
\medbreak
Por lo que se refiere al modelo Random Forest Regression, la autora, \citeA{RandomForestRegression2018} realiz� un art�culo cient�fico sobre este modelo, en donde indica que es un algoritmo de var�as t�cnicas de predicciones (i.e. regresiones y �rboles de decisiones),cabe mencionar que el entrenamiento se basa en la t�cnica llamada, Boostrap Aggregation, que consta en entrenar cada �rbol de decisi�n a partir de las variables de an�lisis, tal como se observa en la figura \ref{fig:RandomForestRegression}:
\begin{figure}[H]
	\caption{T�cnica Random Forest Regression}
	\label{fig:RandomForestRegression}
	\centering
	\includegraphics[width=400px,height=170px]{graphics/random-forest-regression.png} \\
	\textbf{Fuente:} Realizado por el autor de tesis
\end{figure}
\medbreak
Finalmente, el modelo continuo es recomendable utilizarlo para identificar movimientos din�micos (e.g. Sentadillas, abdominales, saltos, entre otros movimientos m�s...), tal como se observa en la siguiente figura:
\begin{figure}[H]
	\caption{Etiquetas de movimientos din�micos}
	\label{fig:modeloContinuo}
	\centering
	\includegraphics[width=400px,height=250px]{graphics/modelo-continuo.png} \\
	\textbf{Fuente:} Realizado por el autor de tesis
\end{figure}
En la figura \ref{fig:modeloContinuo}, se esta analizando el movimiento de vuelo, la cual se conforma de tres movimientos est�ticos: Brazos abajos (i.e. Figura \ref{fig:modeloContinuo}.B), brazos al medio (i.e. Figura \ref{fig:modeloContinuo}.C)  y brazos arriba (i.e. Figura \ref{fig:modeloContinuo}.D),  por lo tanto para analizar y detectar el movimiento din�mico, se etiqueta con un valor decimal a cada movimiento est�tico, tal como se observa en la figura, el primer paso del movimiento din�mico esta entre el valor de 0.00 a 0.33. 
\subparagraph{Skeletal Tracking}\mbox{} \\
En el an�lisis y estudio de los c�digo de fuente de SDK del Kinect \cite{hernandez2013analisis}, los autores realizaron un an�lisis del seguimiento del esqueleto, en la cual observaron que genera la figura humana a trav�s de 25 puntos que representa las principales articulaciones del cuerpo, tal como se muestra en la figura \ref{fig:jointsKinect}:
\begin{figure}[H]
	\caption{Seguimiento de uniones del Kinect}
	\label{fig:jointsKinect}
	\centering
	\includegraphics[width=360px,height=450px]{graphics/jointKinect.png} \\
	\textbf{Fuente:} \cite[]{rocha2015kinect}
\end{figure}
As� mismo los autores, \citeA{hernandez2013analisis}, indican que el seguimiento del esqueleto asocia los par�metros del cuerpo humano (e.g. Extremidades superiores, articulaciones, gestos...), dichos par�metros son utilizado por un SkeletonFrame, cuya finalidad es reconocer el borde del cuerpo humano y posteriormente reconocer cada parte del esqueleto humano, a partir del SkeletonData, tal como se muestra en la figura \ref{fig:skeletanTracking}:
\begin{figure}[H]
	\caption{Arquitectura del seguimiento del esqueleto}
	\label{fig:skeletanTracking}
	\centering
	\includegraphics[width=450px,height=200px]{graphics/SkeletanTracking.png} \\
	\textbf{Fuente:} Elaborado por el autor de tesis
\end{figure}
En la figura \ref{fig:skeletanTracking} se observa que el seguimiento de esqueleto esta conformado por 6 pasos: (1) como primer paso el Kinect escanea el ambiente constantemente, (2) posteriormente se crea el mapa de profundidad,(3) esto permitir�  detectar el suelo y separar los objetos, con el fin de objetivo de identificar el contorno humano de cada jugador, (4) luego por cada jugador se le asigna un identificador y clasifica las partes del cuerpo humano (e.g. Cabeza, brazos, manos, piernas, etc...), (5) seguidamente a cada parte del cuerpo humano se le asigna una uni�n  y (6) finalmente unifica cada uni�n en el orden correspondiente, tal como se observa en la figura \ref{fig:jointsKinect}.
\medbreak
En cuanto las uniones (Joints), los autores,  \citeA{hernandez2013analisis}, menciona que cada uni�n se encuentra en un sistema diestro de coordenada en donde cada eje (i.e. X, Y, Z) representa la distancia (En metros) del Kinect al Joint, tal como se representa en la figura \ref{fig:CoordenadaJoint}:
\begin{figure}[H]
	\caption{Sistema de coordenada de la uni�n (Joint)}
	\label{fig:CoordenadaJoint}
	\centering
	\includegraphics[width=450px,height=190px]{graphics/PartsToJoin.png} \\
	\textbf{Fuente:} Elaborado por el autor de tesis
\end{figure}