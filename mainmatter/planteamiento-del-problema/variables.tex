\section{VARIABLES} \label{vr}
\subsection{VARIABLES DEPENDIENTES} \label{vr:d}
\begin{enumerate}
   \item[A.] Entorno
   \begin{enumerate}
       \item[A.A.] Distancia m\'inima de profundidad entre la persona y el Kinect -i.e. Eje z-.
       \item[A.B.] Distancia m\'axima de profundidad entre la persona y el Kinect -i.e. Eje z-.
   \end{enumerate}
   \item[B.] Kinect
   \begin{enumerate}
      \item[B.A] Articulaciones
       \begin{enumerate}
	       \item[B.A.A.] Horizontal -i.e. Eje x-.
       	   \item[B.A.B.] Vertical -i.e. Eje y-.
       \end{enumerate}
       \item[B.B] Tiempo de captura de la imagen.
   \end{enumerate}
   \item[C.] Habilidades f\'isicas
   \begin{enumerate}
   		\item[C.A.] Flexibilidad.
   		\begin{enumerate}
   			\item[C.A.A.] \'Angulo de movimiento por articulaci\'on.
   		\end{enumerate}
   \end{enumerate}  
   \item[D.] Actividad F\'isica
   \begin{enumerate}   		
   		\item[D.A.] Series de movimientos funcionales.
   	    \item[D.B.] Repeticiones de movimientos funcionales.
   	    \item[D.C.] Entrenamiento Interval\'ado  controlado-
   	    \begin{enumerate}
   	    	\item[D.C.A.] Tiempo de descanso.
   	    	\item[D.C.B.] Tiempo de trabajo.
   	    \end{enumerate}  
   \end{enumerate}  
\end{enumerate}
\subsubsection{Definici\'on de variables dependientes}
\begin{table}[H]
\begin{center}
\caption{Descripci\'on operacional y conceptual de las variables dependientes}
\label{tab:defVarDep}
\begin{tabular}{|c|l|l|}
\hline
\textbf{Enumeracion} & \multicolumn{1}{c|}{\textbf{Definicion operacional}} & \multicolumn{1}{c|}{\textbf{Definicion conceptual}} \\ \hline
\ref{vr:d}.A.A. & \begin{tabular}[c]{@{}l@{}}Distancia de profudidad \\ m\'inima en metros, la cual el Kinect \\ puede detectar correctamente el \\ seguimiento de esqueleto de una persona,\\ para realizar un movimiento funcional.\end{tabular} & \multirow{2}{*}{\begin{tabular}[c]{@{}l@{}}De acuerdo a la secci\'on \ref{mt:cam:mer},\\ dicha distancia \\ es calculada a partir del\\ alcance del sensor 3D de\\ una c\'amara con RGB-D.\\ As\'i mismo el valor esta\\ comprendido entre 1.3 a\\ 3.5 metros (ver Tabla \ref{tab:RGBD}).\end{tabular}} \\ \cline{1-2}
\ref{vr:d}.A.B. & \begin{tabular}[c]{@{}l@{}}Distancia de profudidad \\ m\'axima en metros, la cual el Kinect \\ puede detectar correctamente el \\ seguimiento de esqueleto de una persona,\\ para realizar un movimiento funcional.\end{tabular} &  \\ \hline
\ref{vr:d}.B.A.A. & \begin{tabular}[c]{@{}l@{}}Distancia horizontal en p\'ixeles, entre\\ el Kinect y una articulaci\'on de una \\ persona.\\ \\ \\ \\ \end{tabular} & \multirow{2}{*}{\begin{tabular}[c]{@{}l@{}}Seg\'un la secci\'on \ref{mt:cam:kin}.\ref{mt:cam:kin:st},\\ cada articulaci\'on del \\ esqueleto esta dado por el \\ Skeleton Data, la cual es \\ mapeado por un espacio de\\ profundidad, para describir\\ la ubicaci\'on en 2D -i.e \\ Altura y anchura- por medio\\ del Skeleton Frame (ver \\ Figura  \ref{fig:skeletanTracking}).\end{tabular}} \\ \cline{1-2}
\ref{vr:d}.B.A.B. & \begin{tabular}[c]{@{}l@{}}Distancia vertical en p\'ixeles, entre\\ el Kinect y una articulaci\'on de una \\ persona. \\ \\ \\ \\ \\ \end{tabular} &  \\ \hline
\ref{vr:d}.B.B. & \begin{tabular}[c]{@{}l@{}}Tiempo en segundos,  en donde el Kinect\\ dibuja los frames del seguimiento del\\ esqueleto -i.e. 0.017 \\ segundos por frame-.\end{tabular} & \begin{tabular}[c]{@{}l@{}}Concorde a la secci\'on \ref{mt:cam:kin}.\ref{mt:cam:kin:ks},\\ el seguimiento de esqueleto son\\ necesarios los NUIs de: Profundidad\\ e infrarroja (ver Figura \ref{fig:VennStreaming}). Por\\ lo tanto el seguimiento de esqueleto\\ corre a 60 fps  -i.e. 30 fps de \\ resoluci\'on de 3D y 30 fps de \\ resoluci\'on  RGB-. (ver Tabla \ref{tab:RGBD}).\end{tabular} \\ \hline
\end{tabular}
\end{center}
\textbf{Fuente:} Elaborada por el autor de tesis
\end{table}
\begin{table}[H]
\begin{center}
\caption{Continuacion Descripci\'on operacional y conceptual de las variables dependientes}
\label{tab:defVarDep2}
\begin{tabular}{|c|l|l|}
\hline
\textbf{Enumeracion} & \multicolumn{1}{c|}{\textbf{Definicion operacional}} & \multicolumn{1}{c|}{\textbf{Definicion conceptual}} \\ \hline
\ref{vr:d}.C.A.A. & \begin{tabular}[c]{@{}l@{}}\'Angulo en grado sexagesimal, que \\ determina el arco de movimiento de \\ cada articulacion del cuerpo humano.\end{tabular} & \begin{tabular}[c]{@{}l@{}}Conforme a la seccion \ref{mt:mf:var} la,\\ flexibilidad mide el rango del\\ movimiento muscular, de\\ acuerdo a los arcos de movilidad \\ del cuerpo humano\\ (ver figuras: \ref{fig:ArcosdeMovilidad} y \ref{fig:ArcosdeMovilidad2}).\end{tabular} \\ \hline
\ref{vr:d}.D.A. & \begin{tabular}[c]{@{}l@{}}Unidad que mide las repeticiones de\\ un movimiento de funcional . \\ \\ \\ \\\end{tabular} & \multirow{2}{*}{\begin{tabular}[c]{@{}l@{}}De acuerdo a la secci\'on \ref{mt:af:var}, al \\ momento de realizar actividad f\'isica,\\ la persona realiza ciertas cantidades \\ de series de movimientos f\'isico,\\ conformado por una cantidad de \\ repeticiones -i.e.  movimientos \\ repetitivos-.\end{tabular}} \\ \cline{1-2}
\ref{vr:d}.D.B. & \begin{tabular}[c]{@{}l@{}}Unidad que mide el lapso del\\ paso inicial y final de un movimiento\\ funcional, de manera repetitiva. \\ \\ \end{tabular} &  \\ \hline
\ref{vr:d}.D.C.A. & \begin{tabular}[c]{@{}l@{}}Tiempo en segundos,  la cual pausa\\ la actividad de seguimiento esqueleto. \\ \\ \end{tabular} & \multirow{2}{*}{\begin{tabular}[c]{@{}l@{}}Segun la secci\'on \ref{mt:mf:rut}.\ref{mt:mf:rut:hiit}, estas\\ variables son necesarias para realizar\\ el entrenamiento intervalado \\ controlado, tabata. En donde el tiempo\\ de trabajo debe ser mayor o igual al \\ tiempo de descanso.\end{tabular}} \\ \cline{1-2}
\ref{vr:d}.D.C.B & \begin{tabular}[c]{@{}l@{}}Tiempo en segundos, la cual esta \\ activo la actividad del seguimiento del\\ esqueleto \\ \\.\end{tabular} &  \\ \hline
\end{tabular}
\end{center}
\textbf{Fuente:} Elaborada por el autor de tesis
\end{table}
\subsection{VARIABLES INDEPENDIENTES}
\begin{enumerate}
   \item[A.] Entorno
    \begin{enumerate}
       \item[A.A.] Altura del Kinect -i.e. Eje y-
       \item[A.B.] Punto de an\'alisis -i.e. Centro de equilibrio-.
   \end{enumerate}
   \item[B.] Habilidades f\'isicas
   \begin{enumerate}
   		\item[B.A.] Precisi\'on.
   		\begin{enumerate}
   			\item[B.A.A] Clasificaci\'on por cada paso del movimiento funcional -i.e. Correcto o incorrecto-.
   		\end{enumerate}
   		\item[B.B.] Velocidad.
   		\begin{enumerate}
   			\item[B.B.A.] Tiempo por repetici\'on.
   			\item[B.B.B.] Tiempo por serie.
   		\end{enumerate}
   		\item[B.C.] Agilidad.
   		\begin{enumerate}
   			\item[B.C.A] Tiempo por cada paso del movimiento funcional.
   		\end{enumerate}
   		\item[B.D.] Potencia.
   		\begin{enumerate}
   			\item[B.D.A] Cantidad m\'axima de series por repetici\'on.
   		\end{enumerate}
  \end{enumerate}
  \item[C.] Visual Gesture Builder
  \begin{enumerate}
  	\item[C.A.] Factor discreto.
  	\item[C.A.] Factor continuo.
  \end{enumerate}
\end{enumerate}
