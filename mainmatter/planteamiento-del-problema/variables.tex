\section{Variables} \label{vr}
\subsection{Variables dependientes} \label{vr:dep}
\begin{table}[H]
\begin{center}
\caption{Definiciones de variables dependientes}
\label{tab:defDep}
\begin{tabular}{|l|l|l|}
\hline
\multirow{2}{*}{Nombre} & \multicolumn{2}{c|}{Definiciones} \\ \cline{2-3} 
 & Operacional & Conceptual \\ \hline
\begin{tabular}[c]{@{}l@{}}Articulaciones \\ del\\ movimiento\end{tabular} & \begin{tabular}[c]{@{}l@{}}Son aquellas articulaciones\\ que interviene para realizar\\ el movimiento -i.e. Arcos\\ de movilidad-.\end{tabular} & \begin{tabular}[c]{@{}l@{}}Conjunto de \'indices enteros\\ que representa las articulaciones y\\ arcos de movilidad \\ del seguimiento del esqueleto\end{tabular} \\ \hline
\begin{tabular}[c]{@{}l@{}} N\'umeros \\de pasos \end{tabular} & \begin{tabular}[c]{@{}l@{}} Cantidad de pasos de an\'alisis \\ que tiene el movimiento \end{tabular} & \begin{tabular}[c]{@{}l@{}} De acuerdo a las variables del \\ movimiento, los pasos ayudan en la \\ coordinaci\'on del movimiento \end{tabular} \\ \hline
\begin{tabular}[c]{@{}l@{}} Articulaci\'on \\ de an\'alisis \end{tabular} & \begin{tabular}[c]{@{}l@{}} Articulaci\'on que ayuda \\ a determinar las distancias \\ m\'aximas y m\'inimas \\ de profundidad \end{tabular} & \begin{tabular}[c]{@{}l@{}} Articulaci\'on del seguimiento \\ del esqueleto \end{tabular} \\ \hline
\begin{tabular}[c]{@{}l@{}} Altura del \\ usuario \end{tabular} & \begin{tabular}[c]{@{}l@{}} Variable \\ calculada a partir del \\ seguimiento del esqueleto \end{tabular} & \begin{tabular}[c]{@{}l@{}} Variable que puede afectar \\ en los movimientos cinem\'aticos, \\ dado que altera  en las dimensiones \\ del seguimiento del esqueleto \end{tabular} \\ \hline
\begin{tabular}[c]{@{}l@{}} Altura del Kinect \\  respecto al suelo \end{tabular} & \begin{tabular}[c]{@{}l@{}} Variable que permanece constante \\ por movimiento, adem\'as de ser \\ el punto central de \\ captura de datos del Kinect \end{tabular} & \begin{tabular}[c]{@{}l@{}} La transmisi\'on de datos del Kinect \\ reduce el error del \\ seguimiento del esqueleto, \\ dado que mejora el \\ campo de visi\'on del sensor \end{tabular} \\ \hline
\begin{tabular}[c]{@{}l@{}} Tiempo de captura \\ de dato \end{tabular} & \begin{tabular}[c]{@{}l@{}} Unidad de tiempo \\ con respecto al \\ movimiento de an\'alisis \end{tabular} & \begin{tabular}[c]{@{}l@{}} De acuerdo a la resoluci\'on de 3D \\ y del color, \\ el Kinect percibe un total \\ de 30 fotogramas por segundos, \\ es decir que cada dato \\ se captura cada 0.033 segundos. \end{tabular} \\ \hline
\begin{tabular}[c]{@{}l@{}} Etiquetas \end{tabular} & \begin{tabular}[c]{@{}l@{}} Valor que identifica \\ el paso de un movimiento. \end{tabular} & \begin{tabular}[c]{@{}l@{}} Identifica la \\ coordinaci\'on de un movimiento \end{tabular} \\ \hline
\begin{tabular}[c]{@{}l@{}} Rango m\'aximo \\ de identificaci\'on \end{tabular} & \begin{tabular}[c]{@{}l@{}} Rango que identifica el \\ paso de un movimiento \end{tabular} & \begin{tabular}[c]{@{}l@{}} Equivale a la precisi\'on \\ de un movimiento \end{tabular} \\ \hline
\end{tabular}
\end{center}
\textbf{Fuente:} Elaborado por el autor de tesis
\end{table}
\subsection{Variables independientes} \label{vr:indep}

\begin{table}[H]
\begin{center}
\caption{Definiciones de variables Independientes}
\label{tab:defInde}
\begin{tabular}{|l|l|l|}
\hline
\multirow{2}{*}{Nombre} & \multicolumn{2}{c|}{Definiciones} \\ \cline{2-3} 
 & Operacional & Conceptual \\ \hline
\begin{tabular}[c]{@{}l@{}} Distancia m\'inima \\ y m\'axima de profundidad \end{tabular} & \begin{tabular}[c]{@{}l@{}} Rango ideal para detectar \\ el seguimiento del esqueleto \end{tabular} & \begin{tabular}[c]{@{}l@{}} Variables que se determinan \\ a partir del campo de visi\'on \\ y alcance de profundidad \\ del sensor Kinect \end{tabular} \\ \hline
\begin{tabular}[c]{@{}l@{}} Desplazamiento de \\ articulaciones \end{tabular} & \begin{tabular}[c]{@{}l@{}} Muestra el movimiento \\ de cada articulaci\'on \end{tabular} & \begin{tabular}[c]{@{}l@{}} Variable del movimiento cinem\'atico \end{tabular} \\ \hline
\begin{tabular}[c]{@{}l@{}} Factor del movimiento \end{tabular} & \begin{tabular}[c]{@{}l@{}} Valor que identifica \\ el seguimiento del esqueleto \end{tabular} & \begin{tabular}[c]{@{}l@{}} Variable calculada a \\ partir del algoritmo \\ bosques de regresiones aleatorios \end{tabular} \\ \hline
\begin{tabular}[c]{@{}l@{}} Repetici\'on \end{tabular} & \begin{tabular}[c]{@{}l@{}} Variable que almacena toda \\ la informaci\'on \\ del movimiento \end{tabular} & \begin{tabular}[c]{@{}l@{}} Variable de la Organizaci\'on Mundial  \\ de la Salud que se utiliza  \\ para contabilizar la \\ actividad f\'isica \end{tabular} \\ \hline
\end{tabular}
\end{center}
\textbf{Fuente:} Elaborado por el autor de tesis
\end{table}