\section{APORTES}
Hoy en d\'ia, la inactividad f\'isica es un problema mundial debido que el 60\% de la poblaci\'on mundial no realiza la actividad f\'isica necesaria para obtener beneficios para la salud, y por lo tanto se puede desarrollar enfermedades cr\'onicas no transmisibles, como la diab\'etes. De tal modo que la organizaci\'on Mundial de la salud crea la iniciativa de "Por tu salud, muev\'ete". \cite{orgSaludAF}
\medbreak
Por otro lado, una tercera parte de la poblaci\'on guatemalteca no cumple con las recomendaciones de la organizaci\'on mundial de la salud -i.e. Realizar actividad f\'isica al menos 30 minutos al d\'ia-. Por estos motivos, Guatemala tiene un alto porcentaje de sobrepeso y obesidad , por lo tanto el Ministerio de Salud p\'ublica y Asistencia social impulsa el lema "Salud para todos y todas, en todas partes" \cite{minSaludPub}
\medbreak
A partir de estas iniciativas, el presente proyecto aporta un software que motive a los guatemaltecos a realizar actividad f\'isica de alta intensidad, con la ayuda de dos \'areas de estudios:
\begin{itemize}
	\item \textbf{\gls{visArt}:} El estudio propone un procedimiento ordenado que permite detectar el movimiento del seguimiento de esqueleto de una persona, a partir de distintos patrones -i.e. Pasos que se debe seguir para ejecutar un movimiento repetitivo-. Por otra parte, la investigaci\'on brinda informaci\'on sobre el funcionamiento del sensor Kinect y sus respectivas herramientas, las cuales ayudaron a crear una base de datos de reconocimiento de  movimiento usando la tecnolog\'ia de m\'aquinas de aprendizajes -i.e. Random Forest Regression-.
	\item \textbf{Salud y deporte para el control del sedentarismo:} La investigaci\'on aporta un software que mide las repeticiones de un movimiento, la cual se configura a partir de variables asignada de un tabata -i.e. Series de tiempos de trabajos y descansos-.
\end{itemize}