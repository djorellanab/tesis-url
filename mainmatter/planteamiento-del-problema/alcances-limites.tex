\section{Alcances y limitaciones}
\begin{itemize}
\item Se trabajar\'a con los equipos deportivos que entrenan dentro de la Universidad Rafael Land\'ivar.
\item Se instalar\'a el prototipo de toma de datos del proyecto, en los lugares de entrenamiento de cada equipo deportivo.
\item En cada lugar de entrenamiento debe haber una fuente de energ\'ia para conectar la computadora port\'atil y el sensor Kinect.
\item Cada deporte de la Universidad Rafael Land\'ivar, trabajan  3 d\'ias de entrenamiento competitivo -e.g. Combates, partidos, rutinas- y un d\'ia de entrenamiento de aprendizaje -e.g. T\'ecnicas, movimientos, dominio-. Por lo tanto, se recolectar\'a los datos durante el entrenamiento de aprendizaje de cada deporte.
\item Se utilizar\'a un sensor Kinect, la cual se posicionar\'a de manera que genere completamente el seguimiento del esqueleto.
\item Por cada deporte seleccionado, se analizar\'a un movimiento que no salga del campo de visi\'on del sensor Kinect.
\item Antes de la captura de datos, el atleta debe realizar el calentamiento que realiza constantemente.
\item Durante toda las etapas, el atleta debe estar en un entorno adecuado y as\'i mismo con la vestimenta correcta -e.g. Ropa deportiva-.
\item Debe estar presente el entrenador o profesional, durante la captura de datos.
\item  La captura de datos se realizar\'a de manera individual -i.e. Atleta por atleta-.
\item  La recolecci\'on de datos se centra en el seguimiento de esqueleto del atleta.
\item La validaci\'on de datos se realizar\'a con los atletas que no participar\'an en la etapa de recolecci\'on de datos.
\item La actividad f\'isica se contabilizar\'a por medio de las repeticiones de un  movimiento, ejecutados durante una rutina tabata.
\item Las rutinas tabata ser\'an realizadas conjuntamente por el entrenador y el investigador.
\item Las aplicaciones que implemente el seguimiento del esqueleto, ser\'an desarrolladas para el sistema operativo windows 10.
\item \'Unicamente se visualizar\'a los resultados de la rutina actual (no se llevar\'an los registro hist\'oricos de resultado de un atleta).
\end{itemize}