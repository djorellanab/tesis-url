\documentclass[12pt,letterpaper]{book}
\usepackage{mathptmx}
\usepackage[latin1]{inputenc}
\usepackage[spanish,es-tabla]{babel}
\usepackage{amsmath}
\usepackage{amssymb, amsfonts, latexsym, cancel}
\usepackage{transparent}
\usepackage{eso-pic,graphicx}
\usepackage{epstopdf}
\usepackage{float}
\usepackage{subfigure}
\usepackage{array}
\usepackage{longtable}
\usepackage[left=2cm,right=2cm,top=2cm,bottom=2cm]{geometry}
\usepackage{bm}
\usepackage{appendix}
\usepackage{subfigure}
\usepackage[subfigure]{tocloft}
\usepackage{multirow} % para las tablas
\usepackage{longtable}
\usepackage{lipsum}
\usepackage[breaklinks=true]{hyperref}
\usepackage[acronym,section=section]{glossaries}
\usepackage{fancyhdr, lastpage}
\usepackage{afterpage}
\usepackage{caption,newfloat}
%%%%%%%%%%%%%%%%%%%%%%%%%%%%%%%%%%%%%%%%%%%%%%%%%%
% Agregar una pagina en blanco
\newcommand\blankpage{%
    \null
    \thispagestyle{empty}%
    \addtocounter{page}{-1}%
    \newpage}
%%%%%%%%%%%%%%%%%%%%%%%%%%%%%%%%%%%%%%%%%%%%%%%%%%
% Modificacion de estilos de encabezados y pie de pagina
\fancypagestyle{plain}{
  \fancyhf{}% Clear header/footer
  \fancyfoot[R]{\thepage}% Right footer
  \renewcommand{\headrulewidth}{0.0pt}%
}
\pagestyle{plain}% Set page style to plain.
%%%%%%%%%%%%%%%%%%%%%%%%%%%%%%%%%%%%%%%%%%%%%%%%%%
% Modificar los niveles de nuemeracion
\setcounter{secnumdepth}{4}
%%%%%%%%%%%%%%%%%%%%%%%%%%%%%%%%%%%%%%%%%%%%%%%%%%
% Modificacion de sangria
\parindent=0cm 
%%%%%%%%%%%%%%%%%%%%%%%%%%%%%%%%%%%%%%%%%%%%%%%%%%
% configuracion de anexos
\renewcommand{\appendixname}{Anexos}
\renewcommand{\appendixtocname}{Anexos}
\renewcommand{\appendixpagename}{Anexos}
\begin{document}
\frontmatter
\begin{titlepage}
\begin{center}
\AddToShipoutPictureBG*{\transparent{0.2}\includegraphics[width=\paperwidth,height=\paperheight]{graphics/logo.jpg}}
{\LARGE \textbf{UNIVERSIDAD RAFAEL LAND�VAR}}\\[0.1cm]
{\normalsize FACULTAD DE INGENIER�A}\\[0.1cm]
{\normalsize DEPARTAMENTO DE INGENIER�A EN INFORM�TICA Y SISTEMAS}\\[4cm]
{\huge \textbf{"M�quina de aprendizaje para la detecci�n de los pasos que se requiere para realizar un movimiento funcional mediante la utilizaci�n de una c�mara con sensor de profundidad"}}\\[0.1cm]
{\LARGE PROYECTO DE INGENIER�A}
\vfill
DIEGO JOS� ORELLANA BOJORQUEZ\\[0.1cm]
CARN� 10101-14\\[2cm]
Guatemala, Octubre de 2019\\[0.1cm]
Campus Central
\newpage
\end{center}
\end{titlepage}
\begin{center}
\AddToShipoutPictureBG*{\transparent{0.2}\includegraphics[width=\paperwidth,height=\paperheight]{graphics/logo.jpg}}
{\LARGE \textbf{UNIVERSIDAD RAFAEL LAND�VAR}}\\[0.1cm]
{\normalsize FACULTAD DE INGENIER�A}\\[0.1cm]
{\normalsize DEPARTAMENTO DE INGENIER�A EN INFORM�TICA Y SISTEMAS}\\[4cm]
{\huge \textbf{"M�quina de aprendizaje para la detecci�n de los pasos que se requiere para realizar un movimiento funcional mediante la utilizaci�n de una c�mara con sensor de profundidad"}}\\[0.1cm]
{\LARGE PROYECTO DE INGENIER�A}
\vfill
{\LARGE Presentada ante el Consejo de la Facultad de Ingenier�a}
\vfill
{\LARGE Por:}\\[0.1cm]
{\LARGE \textbf{DIEGO JOS� ORELLANA BOJORQUEZ}}
\vfill
Previo a optar el t�tulo de:\\[0.1cm]
Ingeniero en Inform�tica y Sistemas
\vfill
En el grado acad�mico de:\\[0.1cm]
Licenciado
\vfill
Guatemala, Octubre de 2019\\[0.1cm]
Campus Central
\end{center}
%%%%%%%%%%%%%%%%%%%%%%%%%%%%%%%%%%%%%%%%%%%%%%%%%
% notificaci�n
\afterpage{\blankpage}
\newpage
\includegraphics[width=18cm,height=22cm]{graphics/notificacion-tesis.jpg}
\afterpage{\blankpage}
\newpage
\includegraphics[width=18cm,height=22cm]{graphics/notificacion-tesis2.jpg}
%%%%%%%%%%%%%%%%%%%%%%%%%%%%%%%%%%%%%%%%%%%%%%%%%
\afterpage{\blankpage}
\newpage
{\LARGE \textbf{Agradecimientos}}\\[2cm]
\begin{itemize}
\item A Dios, por ser alguien que me escucha en todo momento.
\item A mi mam\'a, gracias a ella he llegado tan lejos, adem\'as de estar siempre conmigo en las buenas y en las malas. 
\item A mis hermanos, por tenerme paciencia y confiar en m\'i en todo momento.
\item A mi pap\'a, por acompa\~narme siempre.
\item A mis amigos de la universidad, por ser una gran promoci\'on unida y apoyarnos en todo momento.
\item A mis amigos del colegio, por mantener nuestra amistad y vernos crecer.
\item Ingenieros Stanly Bola\~nos y Victor Orozco, por apoyarme en todo el proceso de trabajo de investigaci\'on.
\item Departamento de deportes de la Universidad Rafael Land\'ivar, por confiar en mi proyecto de ingenier\'ia.
\end{itemize}
\begin{center}
Diego Orellana.
\end{center}
\newpage
\tableofcontents.
\listoffigures.
\listoftables.
\listofcharts.
\listofformulas.
\listofcodes.
\mainmatter
%index
\chapter{INTRODUCCI�N}
\section{LO ESCRITO SOBRE EL TEMA}
\section{MARCO TE�RICO}
\subsection{C\'AMARA CON SENSOR DE PROFUNDIDAD}
Seg�n el estudio, On the performance of the Intel SR300 depth camera: metrological and critical characterization \cite{carfagni2017performance}, menciona que unas de las principales funciones de las \acrfull{RGBD} es adquirir y procesar datos en \acrfull{TRESD}, estas c�maras se han utilizado en el sector industrial y acad�mico en aplicaciones tales como: La localizaci�n y mapeo simult�neos, reconocimiento de posiciones y gestos e ingenier�a inversa. \\
Por otro lado el estudio, RGB-D mapping: Using Kinect-style depth cameras for dense 3D modeling of indoor environments   \cite{carfagni2017performance}, indica que las c�maras con sensor de profundidad capturan p�xeles de informaci�n de im�genes (RGB) y de profundidad, tal como se muestra en la siguiente figura:
\begin{figure}[h]
	\caption{Captura de datos de una c�mara con sensor de profundidad}
	\label{fig:RGBD}
	\centering
	\includegraphics[]{graphics/RGB-D.PNG} \\
	\textbf{Fuente:} Tomado por el autor de tesis
\end{figure}  \\
La figura \ref{fig:RGBD}, fue capturado por el dispositivo, Kinect de XBox One, a una velocidad de 28 \acrfull{FPS}, cabe mencionar que los p�xeles  blancos de la imagen de la derecha, no se puede determinar el valor de profundidad, debido que falta el an�lisis de distancia, angulo relativo de la superficie, material de superficie, y otras variable m�s que se observar� en el presente trabajo.
\section{TRABAJOS RELACIONADOS}
\chapter{PLANTEAMIENTO DEL PROBLEMA}
\section{OBJETIVOS}
\subsection{OBJETIVO GENERAL}
\subsection{OBJETIVOS ESPEC�FICOS}
\section{HIP�TESIS}
\section{VARIABLES}
\subsection{VARIABLES DEPENDIENTES}
\subsection{VARIABLES INDEPENDIENTES}
\section{DEFINICI�N DE LAS VARIABLES}
\subsection{DEFINICI�N CONCEPTUAL}
\subsection{DEFINICI�N OPERACIONAL}
\section{ALCANCES}
\section{L�MITES}
\section{APORTE}
\chapter{M�TODO}
\section{SUJETOS}
\subsection{PRIMER TIPO}
\subsection{SEGUNDO TIPO}
\section{UNIDADES DE AN�LISIS}
\section{INSTRUMENTOS}
\section{PROCEDIMIENTO}
\section{DISE�O Y METODOLOG�A ESTAD�STICA}
\subsection{DISE�O EXPERIMENTAL}
\subsubsection{EXPERIMENTOS}
\subsubsection{TRATAMIENTOS Y REPETICIONES EN LOS EXPERIMENTOS}
\chapter{PRESENTACI�N Y AN�LISIS DE RESULTADOS}
\input{mainmatter/discusion}
\section{Conclusiones} \label{ded:con}
\begin{itemize}
\item De acuerdo a los tres deportes estudiados, se puede analizar movimientos que son f\'aciles de hacer dentro el campo de visi\'on del sensor Kinect.
\item Conforme a los tres movimientos analizados, se puede deducir que un movimiento esta compuesto por dos o m\'as pasos, y en cada paso se debe ejecutar  la postura correcta.
\item En relaci\'on a los entrenamientos de los deportes seleccionados, se puede concluir que el calentamiento es una actividad importante para la preparaci\'on de los movimientos que se ejecutan en una rutina.
\item Acorde a las rutinas de captura de datos utilizados en el proyecto,  se puede inferir que la rutina por tiempo, es utilizado en los atletas que puede ejecutar un mayor volumen de repetici\'on en un largo per\'iodo de tiempo (Sin importar la cantidad de repeticiones), mientras que la rutina por repeticiones (Escaleras), se establece el n\'umero de repeticiones que debe ejecutar el deportista.
\item  Con respecto al seguimiento de esqueleto de cada movimiento analizado, se puede deducir que la distancia correcta para que el sensor Kinect pueda detectar al atleta, es de  2.56 metros a 3.99 metros de profundidad. Sin embargo, el sensor debe estar a una altura de 0.70 metros (Respecto al suelo).
\item  En base a las grabaciones (Tomadas por la herramienta, Kinect Studio) durante la etapa de recolecci\'on de datos, se puede concluir que existe tres tipos de sucesos que puede generar datos inservible; la primera consta de las interrupciones de personas u objetos, adem\'as de los eventos inesperado, como la lluvia. As\'i mismo el segundo suceso se relaciona con las fallas del hardware o software que impide visualizar el seguimiento del esqueleto y finalmente el tercer evento consta en la grabaci\'on dentro de un entorno no ideal, pongamos por caso los espacios abiertos y no planos.
\item En relaci\'on a la implementaci\'on de la base de datos de reconocimiento de gesturas y posturas, se puede deducir que el software, Visual Gesture Builder (Herramienta de inteligencia artificial del SDK del Kinect), crea modelos de reconocimiento a partir de la etiquetaci\'on decimal de fotogramas, es decir por cada paso del movimiento analizado esta etiquetado con un n\'umero entre 0 a 1.
\item De acuerdo a los modelos de pron\'ostico de gesturas y  posturas, se puede concluir que el valor resultante de cada modelo se le llama, factor de movimiento, la cual puede ser comparado por el valor etiquetado de las grabaciones de pruebas, con el fin de objetivo de encontrar los errores del pron\'ostico (Diferencia entre factor movimiento y valor etiquetado).
\item Conforme a los errores de pron\'ostico de cada modelo, se determin\'o el error promedio (desviaci\'on media absoluta), as\'i mismo se construy\'o los rangos de confianza de cada paso, con la ayuda de la desviaci\'on t\'ipica del modelo (Ra\'iz del error cuadr\'atico medio).
\item Por lo que se refiere a la interfaz gr\'afica del seguimiento de esqueleto, se ha utilizado la tecnolog\'ia Windows Presentation Foundation, dado  las herramientas de dibujos (Windows media), manejos de temporizadores (Windows Threading) y manejos de los datos del sensor (SDK de Kinect).
\item En cuanto al algoritmo de detecci\'on de repetici\'on de un movimiento, se detecta \'unicamente, si el seguimiento de esqueleto del atleta pasa por todos los pasos del movimiento de manera ordenada y ascendente.
\item Por lo que se refiere a los resultados de los entrenamiento de cada movimiento estudiado, se ha deducido que la rutina Tabata permite contabilizar las repeticiones del movimiento durante los tiempos de trabajos, adem\'as de ser un entrenamiento de alta intensidad con alta metabolizaci\'on de energ\'ia (Obtenci\'on y gasto de mol\'eculas de ATP) en todo el cuerpo, durante los tiempos de trabajo y descanso.
\item De acuerdo a la comprobaci\'on del modelo de reconocimiento de cada movimiento, se ha comprobado y validado cada modelo por medio de la validaci\'on  cruzada, 3-Fold, la cual permite crear distintos modelos de reconocimiento de movimiento a partir de  combinaciones de datos de pruebas y entrenamientos.
\end{itemize}
\section{Recomendaciones} \label{ded:rec}
\begin{itemize}
\item Si se trabaja con un movimiento complejo, se recomienda analizarlo por dos o m\'as movimientos simples (divide y conquistar\'a), por ejemplo para el movimiento complejo burpee, se puede analizar por dos movimientos simples: Lagartija -i.e. push up- y salto.
\item Para aumentar la precisi\'on y disminuir el error del reconocimiento de movimiento, es necesario m\'as datos de  repeticiones del movimiento (realizadas por distintos atletas), lo cual se aconseja ir a distintos grupos deportivos que realizan dicho movimiento -e.g. Universidades, federaciones, colegios, gimnasios-.
\item Se sugiere capturar los datos en un  lugar que tenga un clima agradable -e.g. Sin lluvia o exceso de sol-, con un espacio suficiente para realizar el movimiento sin ninguna interrupci\'on -e.g. Lugares planos-, adem\'as de tener una  ventilaci\'on e iluminaci\'on correcta -e.g. Lugares que no se acumule  el calor-.
\item Es recomendable para el atleta realizar una rutina de calentamiento con la vestimenta adecuada, para preparar el organismo y evitar alguna lesi\'on durante el entrenamiento.
\item Para futuros proyectos se recomienda implementar un modelo que clasifique una repetici\'on como bueno o malo.
\item Es aconsejable que cada grupo deportivo almacene los resultados de sus atletas, para ver si el sedentarismo de cada deportista ha mejorado o empeorado en un lapso de tiempo.
\end{itemize}
\backmatter
\input{backmatter/bibliografia}
\appendix
\clearpage
\addappheadtotoc
\appendixpage
\chapter{GLOSARIO Y ABREVIACIONES}
\printglossary[]
\end{document}