\begin{titlepage}
\begin{center}
\AddToShipoutPictureBG*{\transparent{0.2}\includegraphics[width=\paperwidth,height=\paperheight]{graphics/logo.jpg}}
{\LARGE \textbf{UNIVERSIDAD RAFAEL LAND�VAR}}\\[0.1cm]
{\normalsize FACULTAD DE INGENIER�A}\\[0.1cm]
{\normalsize DEPARTAMENTO DE INGENIER�A EN INFORM�TICA Y SISTEMAS}\\[4cm]
{\huge \textbf{"M�quina de aprendizaje para la detecci�n de los pasos que se requiere para realizar un movimiento funcional mediante la utilizaci�n de una c�mara con sensor de profundidad"}}\\[0.1cm]
{\LARGE PROYECTO DE INGENIER�A}
\vfill
DIEGO JOS� ORELLANA BOJORQUEZ\\[0.1cm]
CARN� 10101-14\\[2cm]
Guatemala, Octubre de 2019\\[0.1cm]
Campus Central
\newpage
\end{center}
\end{titlepage}
\begin{center}
\AddToShipoutPictureBG*{\transparent{0.2}\includegraphics[width=\paperwidth,height=\paperheight]{graphics/logo.jpg}}
{\LARGE \textbf{UNIVERSIDAD RAFAEL LAND�VAR}}\\[0.1cm]
{\normalsize FACULTAD DE INGENIER�A}\\[0.1cm]
{\normalsize DEPARTAMENTO DE INGENIER�A EN INFORM�TICA Y SISTEMAS}\\[4cm]
{\huge \textbf{"M�quina de aprendizaje para la detecci�n de los pasos que se requiere para realizar un movimiento funcional mediante la utilizaci�n de una c�mara con sensor de profundidad"}}\\[0.1cm]
{\LARGE PROYECTO DE INGENIER�A}
\vfill
{\LARGE Presentada ante el Consejo de la Facultad de Ingenier�a}
\vfill
{\LARGE Por:}\\[0.1cm]
{\LARGE \textbf{DIEGO JOS� ORELLANA BOJORQUEZ}}
\vfill
Previo a optar el t�tulo de:\\[0.1cm]
Ingeniero en Inform�tica y Sistemas
\vfill
En el grado acad�mico de:\\[0.1cm]
Licenciado
\vfill
Guatemala, Octubre de 2019\\[0.1cm]
Campus Central
\end{center}