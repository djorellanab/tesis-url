% \newglossaryentry{nombre} 
% {
% name={Nombre},
% description={desc}
% }
 \newglossaryentry{MovUni} 
 {
 name={movimiento unilateral},
 description={Movimiento la cual emplea las articulaciones de un lado del cuerpo humano, ejemplo una patada (Debido que se puede patear del lado derecho o izquierdo)}
 }
 \newglossaryentry{visArt} 
 {
 name={visi\'on artificial},
 description={Campo de la  inteligencia artificial que permite adquirir, procesar y comprender las im\'agenes del mundo real}
 }
 \newglossaryentry{telem} 
 {
 name={telemetr\'ia},
 description={Tecnolog\'ia que permite la medici\'on de magnitudes f\'isicas}
 }

 \newglossaryentry{brad} 
 {
 name={base radial},
 description={Procedimiento de redes neuronales que calculan la salida en funci\'on de la distancia de un punto respecto a un punto central }
 }
 
  \newglossaryentry{sigm} 
 {
 name={sigmoide},
 description={Funci\'on matem\'atica, que se utiliza para suavizar los modelos de aprendizaje, cuya forma es de una S }
 }
 
   \newglossaryentry{gaus} 
 {
 name={gaussiana},
 description={En t\'erminos de estad\'isticas, representa la distribuci\'on normal de un grupo de datos}
 }
 
    \newglossaryentry{arbdec} 
 {
 name={\'arbol de decisi\'on},
 description={Modelo de clasificador que se divide por distintos nodos de decisiones, para llegar a una respuesta (hoja)}
 }
 
    \newglossaryentry{knnia} 
 {
 name={K vecinos pr\'oximos},
 description={M\'etodo clasificador supervisado, que sirve para estimar la funci\'on de densidad respecto a un conjunto de datos cercanos al dato que se desea pronosticar}
 } 
 
     \newglossaryentry{redneu} 
 {
 name={red neuronal},
 description={Paradigma de aprendizaje y procesamiento autom\'atico, cuya estructura esta formado por un sistema de interconexi\'on de neuronas que colaboran para producir una o varias salidas}
 }
 
      \newglossaryentry{bayesian} 
 {
 name={clasificador bayesiano},
 description={Clasificador probabil\'istico fundamento en el teorema de bayes (probabilidad dado uno o m\'as eventos)}
 }
      \newglossaryentry{matcov} 
 {
 name={matriz de covarianza},
 description={Matriz cuadrada que contiene las varianzas y covarianzas asociadas a diferentes variables}
 } 
      \newglossaryentry{elmia} 
 {
 name={m\'aquina de aprendizaje extremo},
 description={M\'aquina que se conforma de redes neuronales, cuya esencia es que la capa oculta de la red se construye con un entrenamiento r\'apido y con poca participaci\'on humana}
 }  
       \newglossaryentry{svmia} 
 {
 name={m\'aquina de soporte vectorial},
 description={M\'etodo clasificador, la cual separa un conjunto de datos mediante un hiperplano de separaci\'on}
 }
 
  \newglossaryentry{Heur} 
 {
 name={heur\'istica},
 description={M\'etodo basado en la experiencia (informaci\'on previa), que puede utilizarse para resolver problemas }
 }

  \newglossaryentry{diseuc} 
 {	
 name={distancia euclidiana},
 description={Distancia entre dos puntos}
 }
 
 
 \newglossaryentry{pixeles} 
 {
 name={p\'ixeles},
 description={Unidades b\'asicas de una imagen que obtiene el valor de un color}
 }
 
 \newglossaryentry{infrarrojo} 
 {
 name={infrarrojo},
 description={Radiaci\'on electromagn\'etica que emite un cuerpo, independiente a que exista otro tipo de luz}
 }
 
  \newglossaryentry{TOF} 
 {
 name={tiempo de vuelo},
 description={T\'ecnica que se emplea para calcular distancia entre objetos}
 }